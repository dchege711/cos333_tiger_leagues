%% Generated by Sphinx.
\def\sphinxdocclass{report}
\documentclass[letterpaper,10pt,english]{sphinxmanual}
\ifdefined\pdfpxdimen
   \let\sphinxpxdimen\pdfpxdimen\else\newdimen\sphinxpxdimen
\fi \sphinxpxdimen=.75bp\relax

\PassOptionsToPackage{warn}{textcomp}
\usepackage[utf8]{inputenc}
\ifdefined\DeclareUnicodeCharacter
% support both utf8 and utf8x syntaxes
\edef\sphinxdqmaybe{\ifdefined\DeclareUnicodeCharacterAsOptional\string"\fi}
  \DeclareUnicodeCharacter{\sphinxdqmaybe00A0}{\nobreakspace}
  \DeclareUnicodeCharacter{\sphinxdqmaybe2500}{\sphinxunichar{2500}}
  \DeclareUnicodeCharacter{\sphinxdqmaybe2502}{\sphinxunichar{2502}}
  \DeclareUnicodeCharacter{\sphinxdqmaybe2514}{\sphinxunichar{2514}}
  \DeclareUnicodeCharacter{\sphinxdqmaybe251C}{\sphinxunichar{251C}}
  \DeclareUnicodeCharacter{\sphinxdqmaybe2572}{\textbackslash}
\fi
\usepackage{cmap}
\usepackage[T1]{fontenc}
\usepackage{amsmath,amssymb,amstext}
\usepackage{babel}
\usepackage{times}
\usepackage[Bjarne]{fncychap}
\usepackage{sphinx}

\fvset{fontsize=\small}
\usepackage{geometry}

% Include hyperref last.
\usepackage{hyperref}
% Fix anchor placement for figures with captions.
\usepackage{hypcap}% it must be loaded after hyperref.
% Set up styles of URL: it should be placed after hyperref.
\urlstyle{same}

\addto\captionsenglish{\renewcommand{\figurename}{Fig.}}
\addto\captionsenglish{\renewcommand{\tablename}{Table}}
\addto\captionsenglish{\renewcommand{\literalblockname}{Listing}}

\addto\captionsenglish{\renewcommand{\literalblockcontinuedname}{continued from previous page}}
\addto\captionsenglish{\renewcommand{\literalblockcontinuesname}{continues on next page}}
\addto\captionsenglish{\renewcommand{\sphinxnonalphabeticalgroupname}{Non-alphabetical}}
\addto\captionsenglish{\renewcommand{\sphinxsymbolsname}{Symbols}}
\addto\captionsenglish{\renewcommand{\sphinxnumbersname}{Numbers}}

\addto\extrasenglish{\def\pageautorefname{page}}

\setcounter{tocdepth}{1}



\title{Tiger Leagues Documentation}
\date{Jan 15, 2019}
\release{1.0}
\author{Chege, Ivy, Rui, Obinna}
\newcommand{\sphinxlogo}{\vbox{}}
\renewcommand{\releasename}{Release}
\makeindex
\begin{document}

\pagestyle{empty}
\sphinxmaketitle
\pagestyle{plain}
\sphinxtableofcontents
\pagestyle{normal}
\phantomsection\label{\detokenize{index::doc}}


The source code for this project lives in \sphinxurl{https://github.com/dchege711/cos333\_tiger\_leagues}
\phantomsection\label{\detokenize{readme:tiger-leagues-overview}}\phantomsection\label{\detokenize{readme:tiger-leagues-overview}}\sphinxhref{https://travis-ci.com/dchege711/cos333\_tiger\_leagues}{\sphinxincludegraphics{{/Users/dchege711/tiger_leagues/docs/doctrees/images/da419489ddbba68d29dd3328da2a8bd6c8ba4638/cos333_tiger_leagues}.svg}}

\chapter{App Overview}
\label{\detokenize{readme:app-overview}}\label{\detokenize{readme::doc}}
Tiger Leagues is a web application hosted at \sphinxurl{https://tiger-leagues.herokuapp.com/}.
It is used by Princeton students that wish to run a league amongst themselves.
The application was created in lieu of managing leagues over Google Docs.
Features include:
\begin{itemize}
\item {} 
Creating leagues and registering members

\item {} 
Scheduling league games

\item {} 
Keeping track of scores and leaderboards

\end{itemize}


\section{Components}
\label{\detokenize{readme:components}}\label{\detokenize{readme:id1}}
\noindent\sphinxincludegraphics{{tiger_leagues_components}.png}


\section{Getting Started}
\label{\detokenize{readme:getting-started}}\label{\detokenize{readme:id2}}
Clone the repository with:

\begin{sphinxVerbatim}[commandchars=\\\{\}]
\PYGZdl{} git clone https://github.com/dchege711/cos333\PYGZus{}tiger\PYGZus{}leagues.git
\PYGZdl{} cd cos333\PYGZus{}tiger\PYGZus{}leagues
\end{sphinxVerbatim}

Install the python packages (preferably in a new virtual environment):

\begin{sphinxVerbatim}[commandchars=\\\{\}]
\PYGZdl{} pip install requirements.txt
\end{sphinxVerbatim}

We modelled the application after this \sphinxhref{http://flask.pocoo.org/docs/1.0/tutorial/}{flask tutorial}.
Installing Flask usually installs \sphinxcode{\sphinxupquote{ItsDangerous v1.0.0}} as a prerequisite.
However, Heroku cannot install v1.0.0. For this reason, \sphinxcode{\sphinxupquote{./requirements.txt}}
was manually updated to have \sphinxcode{\sphinxupquote{ItsDangerous==0.24}}.

We’re using a locally hosted database for development purposes.
We found this \sphinxhref{https://www.codementor.io/engineerapart/getting-started-with-postgresql-on-mac-osx-are8jcopb}{tutorial}
useful when setting up PostgreSQL.

Set the values of the environment variables defined in \sphinxhref{tiger\_leagues/readme.html\#module-tiger\_leagues.config}{config.py}.

Run the application server:

\begin{sphinxVerbatim}[commandchars=\\\{\}]
\PYGZdl{} ./run\PYGZus{}flask\PYGZus{}server
\end{sphinxVerbatim}


\section{Architecture}
\label{\detokenize{readme:architecture}}\label{\detokenize{readme:id3}}
Tiger Leagues uses the \sphinxhref{https://en.wikipedia.org/wiki/Model\%E2\%80\%93view\%E2\%80\%93controller}{Model-View-Controller}
architectural pattern.

\noindent\sphinxincludegraphics{{mvc_architecture}.png}

For more detailed documentation and design decisions made at each level see:
\begin{itemize}
\item {} 
\sphinxhref{tiger\_leagues/models/readme.html}{Models}

\item {} 
\sphinxhref{tiger\_leagues/models/readme.html}{Views}

\item {} 
\sphinxhref{tiger\_leagues/readme.html}{Controllers}

\end{itemize}


\section{Testing}
\label{\detokenize{readme:testing}}\label{\detokenize{readme:id4}}
The tests are defined in the \sphinxcode{\sphinxupquote{tests}} folder at the root of the project. We’re
using \sphinxhref{https://docs.pytest.org/en/latest/}{pytest} for our testing. We have
included a helper bash script (\sphinxcode{\sphinxupquote{run\_tests}}) to run the tests and provide
coverage analysis.

We have also set up Travis CI to test entire builds. The Travis CI
configuration is defined in the \sphinxcode{\sphinxupquote{.travis.yml}} file at the root. We have set
up \sphinxcode{\sphinxupquote{origin/master}} as a protected branch, so make sure Travis CI greenlights
any pull requests.


\section{Generating the Documentation}
\label{\detokenize{readme:generating-the-documentation}}\label{\detokenize{readme:id5}}
The documentation for this project is modelled after \sphinxhref{https://matplotlib.org/sampledoc/index.html}{Matplotlib’s Tutorial} which generates documentation
with \sphinxhref{http://www.sphinx-doc.org/en/master/}{Sphinx}.

\sphinxcode{\sphinxupquote{conf.py}} sets up the settings used by Sphinx. The docs are built from the
.md and .rst files found within this repository.

You generally need to run \sphinxcode{\sphinxupquote{\$ make html}} to build the documentation. The docs
are built inside the \sphinxcode{\sphinxupquote{docs}} folder, so that GitHub Pages can access them and
serve them at \sphinxurl{https://dchege711.github.io/cos333\_tiger\_leagues}


\section{Additional Feature Requests}
\label{\detokenize{readme:additional-feature-requests}}\label{\detokenize{readme:id6}}
The following are remarks made by users. In case you’re looking for features to
implement, this is it Chief!
\begin{itemize}
\item {} 
You should add a feature to upload and share pictures and videos of the
match to brag.

\item {} 
We can have a puskas award (most beautiful goal) for the league at the end of
the tourney

\item {} 
In-depth stats, highlights, something to track playoff probability/scenarios
toward the end would be pretty cool.

\item {} 
Also a schedule for all the games.

\item {} 
We can watch other ppl’s matches right? Would be cool to have an audience

\end{itemize}


\chapter{Models}
\label{\detokenize{tiger_leagues/models/readme:models}}\label{\detokenize{tiger_leagues/models/readme:tiger-leagues-models}}\label{\detokenize{tiger_leagues/models/readme::doc}}
As described on \sphinxhref{https://en.wikipedia.org/wiki/Model\%E2\%80\%93view\%E2\%80\%93controller\#Components}{Wikipedia},
the model is the application’s dynamic data structure, independent of the user
interface. It directly manages the data, logic and rules of the application.
We deviated a bit from the standard MVC architecture by validating our inputs
at this level.

Here is a quick breakdown of where the higher-level application logic is handled:


\begin{savenotes}\sphinxattablestart
\centering
\begin{tabular}[t]{|*{2}{\X{1}{2}|}}
\hline
\sphinxstyletheadfamily 
Model
&\sphinxstyletheadfamily 
Application Logic
\\
\hline
{\hyperref[\detokenize{tiger_leagues/models/readme:module-tiger_leagues.models.league_model}]{\sphinxcrossref{\sphinxcode{\sphinxupquote{tiger\_leagues.models.league\_model}}}}}
&\begin{itemize}
\item {} 
Creating a new league

\item {} 
Recording requests to join a league

\item {} 
Updating league standings

\item {} 
Fetching league standings

\item {} 
Fetching league matches

\item {} 
Fetching player stats

\item {} 
Processing score reports submitted by players

\item {} 
Processing player requests to leave a league

\end{itemize}
\\
\hline
{\hyperref[\detokenize{tiger_leagues/models/readme:module-tiger_leagues.models.admin_model}]{\sphinxcrossref{\sphinxcode{\sphinxupquote{tiger\_leagues.models.admin\_model}}}}}
&\begin{itemize}
\item {} 
Adding/removing players from a league

\item {} 
Allocating league divisions

\item {} 
Processing score reports submitted by admins

\item {} 
Deleting a league

\end{itemize}
\\
\hline
{\hyperref[\detokenize{tiger_leagues/models/readme:module-tiger_leagues.models.user_model}]{\sphinxcrossref{\sphinxcode{\sphinxupquote{tiger\_leagues.models.user\_model}}}}}
&\begin{itemize}
\item {} 
Fetch existing user profile

\item {} 
Update a user’s profile

\item {} 
Post notifications to a user

\item {} 
Read user’s notifications

\end{itemize}
\\
\hline
{\hyperref[\detokenize{tiger_leagues/models/readme:module-tiger_leagues.models.exception}]{\sphinxcrossref{\sphinxcode{\sphinxupquote{tiger\_leagues.models.exception}}}}}
&\begin{itemize}
\item {} 
Raise exceptions caused by errors encountered
when accomplishing any of the above logic

\end{itemize}
\\
\hline
\end{tabular}
\par
\sphinxattableend\end{savenotes}


\section{Design Decisions}
\label{\detokenize{tiger_leagues/models/readme:design-decisions}}\label{\detokenize{tiger_leagues/models/readme:models-design-decisions}}

\subsection{Application Context}
\label{\detokenize{tiger_leagues/models/readme:application-context}}\label{\detokenize{tiger_leagues/models/readme:id1}}
Tiger Leagues can be ran in 3 different contexts. We used environment
variables and {\hyperref[\detokenize{tiger_leagues/models/readme:module-tiger_leagues.models.config}]{\sphinxcrossref{\sphinxcode{\sphinxupquote{tiger\_leagues.models.config}}}}} to switch between the
different contexts:
\begin{itemize}
\item {} 
Development

\end{itemize}

Occurs when Tiger Leagues is being ran locally. We found it convenient to use
a locally hosted database so that the dev can modify its content as they see
fit.
\begin{itemize}
\item {} 
Travis CI

\end{itemize}

Our tests write and read data from the database. We found it convenient to
use a PostgreSQL database that is provisioned by Travis CI. The database is
set up by the \sphinxcode{\sphinxupquote{.travis.yml}} file at the root of the repository.
\begin{itemize}
\item {} 
Heroku (Production)

\end{itemize}

If Tiger Leagues is running on Heroku, we use the database provided by Heroku.


\subsection{League Rankings}
\label{\detokenize{tiger_leagues/models/readme:league-rankings}}\label{\detokenize{tiger_leagues/models/readme:league-standings}}
Initially, we’d compute the rankings of players from the matches table. This
was motivated by reducing redundancy in the database. However, we hypothesized
that users will view the rankings way more frequently than update scores.

We therefore decided to add another table, \sphinxcode{\sphinxupquote{league\_standings}} that contains
the most recent rankings that incorporate all the approved score reports.
Although this creates some redundancy (we could determine the rankings from
the score reports), it allows us to reduce repeated computation.


\subsection{Keeping the User Updated}
\label{\detokenize{tiger_leagues/models/readme:keeping-the-user-updated}}\label{\detokenize{tiger_leagues/models/readme:id2}}
Since the users are not isolated, it’s important to keep a user updated of any
developments that involve them. For instance, a user might get their join
request approved/denied by an admin. Or the score for a given match might have
been reported by the other player and the admin approved of it.

We therefore developed a rudimentary notification system in which we post
relevant updates to a user’s mailbox. The system does not allow for responses.
We leave that for future implementations of Tiger Leagues.


\section{Models Documentation}
\label{\detokenize{tiger_leagues/models/readme:models-documentation}}\label{\detokenize{tiger_leagues/models/readme:id3}}

\subsection{tiger\_leagues.models.db\_model}
\label{\detokenize{tiger_leagues/models/readme:module-tiger_leagues.models.db_model}}\label{\detokenize{tiger_leagues/models/readme:tiger-leagues-models-db-model}}\index{tiger\_leagues.models.db\_model (module)@\spxentry{tiger\_leagues.models.db\_model}\spxextra{module}}
db.py

A wrapper around the database used by the ‘Tiger Leagues’ app
\index{Database (class in tiger\_leagues.models.db\_model)@\spxentry{Database}\spxextra{class in tiger\_leagues.models.db\_model}}

\begin{fulllineitems}
\phantomsection\label{\detokenize{tiger_leagues/models/readme:tiger_leagues.models.db_model.Database}}\pysiglinewithargsret{\sphinxbfcode{\sphinxupquote{class }}\sphinxcode{\sphinxupquote{tiger\_leagues.models.db\_model.}}\sphinxbfcode{\sphinxupquote{Database}}}{\emph{connection\_uri=None}}{}
A wrapper around the database used by the ‘Tiger Leagues’ app.
\begin{quote}\begin{description}
\item[{Parameters}] \leavevmode
\sphinxstyleliteralstrong{\sphinxupquote{connection\_uri}} \textendash{} str

\end{description}\end{quote}

Optional connection string for the database. If \sphinxcode{\sphinxupquote{None}}, this defaults to 
the connection string set in \sphinxcode{\sphinxupquote{config.DATABASE\_URL}}.
\index{disconnect() (tiger\_leagues.models.db\_model.Database method)@\spxentry{disconnect()}\spxextra{tiger\_leagues.models.db\_model.Database method}}

\begin{fulllineitems}
\phantomsection\label{\detokenize{tiger_leagues/models/readme:tiger_leagues.models.db_model.Database.disconnect}}\pysiglinewithargsret{\sphinxbfcode{\sphinxupquote{disconnect}}}{}{}
Close the connection to the database. Should be called before exiting 
the script.

\end{fulllineitems}

\index{execute() (tiger\_leagues.models.db\_model.Database method)@\spxentry{execute()}\spxextra{tiger\_leagues.models.db\_model.Database method}}

\begin{fulllineitems}
\phantomsection\label{\detokenize{tiger_leagues/models/readme:tiger_leagues.models.db_model.Database.execute}}\pysiglinewithargsret{\sphinxbfcode{\sphinxupquote{execute}}}{\emph{statement}, \emph{values=None}, \emph{dynamic\_table\_or\_column\_names=None}, \emph{cursor\_factory=\textless{}class 'psycopg2.extras.DictCursor'\textgreater{}}}{}~\begin{quote}\begin{description}
\item[{Parameters}] \leavevmode
\sphinxstyleliteralstrong{\sphinxupquote{statement}} \textendash{} str

\end{description}\end{quote}

The SQL query to run.
\begin{quote}\begin{description}
\item[{Parameters}] \leavevmode
\sphinxstyleliteralstrong{\sphinxupquote{values}} \textendash{} list

\end{description}\end{quote}

Values that the query’s placeholders should be replaced with
\begin{quote}\begin{description}
\item[{Parameters}] \leavevmode
\sphinxstyleliteralstrong{\sphinxupquote{dynamic\_table\_or\_column\_names}} \textendash{} list

\end{description}\end{quote}

Names of tables/columns that should be substituted into the SQL statement
\begin{quote}\begin{description}
\item[{Parameters}] \leavevmode
\sphinxstyleliteralstrong{\sphinxupquote{cursor\_factory}} \textendash{} psycopg2.extensions.cursor

\end{description}\end{quote}

The type of object that should be generated by calls to the \sphinxcode{\sphinxupquote{cursor()}} 
method.
\begin{quote}\begin{description}
\item[{Returns}] \leavevmode
\sphinxcode{\sphinxupquote{cursor}}

\end{description}\end{quote}

The cursor after after executing the SQL query
\begin{quote}\begin{description}
\item[{Raise}] \leavevmode
\sphinxcode{\sphinxupquote{psycopg2.errors}}

\end{description}\end{quote}

If the SQL transaction fails, the transaction is rolled back. The most 
recently executed query is printed to \sphinxcode{\sphinxupquote{sys.stderr}}. The error is then 
raised.

\end{fulllineitems}

\index{execute\_many() (tiger\_leagues.models.db\_model.Database method)@\spxentry{execute\_many()}\spxextra{tiger\_leagues.models.db\_model.Database method}}

\begin{fulllineitems}
\phantomsection\label{\detokenize{tiger_leagues/models/readme:tiger_leagues.models.db_model.Database.execute_many}}\pysiglinewithargsret{\sphinxbfcode{\sphinxupquote{execute\_many}}}{\emph{sql\_query}, \emph{values}, \emph{dynamic\_table\_or\_column\_names=None}, \emph{cursor\_factory=\textless{}class 'psycopg2.extras.DictCursor'\textgreater{}}}{}
Execute many related SQL queries, e.g. update several rows of a table.
\begin{quote}\begin{description}
\item[{Parameters}] \leavevmode
\sphinxstyleliteralstrong{\sphinxupquote{sql\_query}} \textendash{} str

\end{description}\end{quote}

The SQL query to run. It must contain a single \sphinxtitleref{\%s} placeholder
\begin{quote}\begin{description}
\item[{Parameters}] \leavevmode
\sphinxstyleliteralstrong{\sphinxupquote{values}} \textendash{} iterable

\end{description}\end{quote}

Each item should be a value that can be substituted when composing a 
SQL query
\begin{quote}\begin{description}
\item[{Parameters}] \leavevmode
\sphinxstyleliteralstrong{\sphinxupquote{dynamic\_table\_or\_column\_names}} \textendash{} list

\end{description}\end{quote}

Names of tables/columns that should be substituted into the SQL statement
\begin{quote}\begin{description}
\item[{Parameters}] \leavevmode
\sphinxstyleliteralstrong{\sphinxupquote{cursor\_factory}} \textendash{} psycopg2.extensions.cursor

\end{description}\end{quote}

The type of object that should be generated by calls to the \sphinxcode{\sphinxupquote{cursor()}} 
method.
\begin{quote}\begin{description}
\item[{Returns}] \leavevmode
\sphinxcode{\sphinxupquote{cursor}}

\end{description}\end{quote}

The cursor after after executing the SQL query
\begin{quote}\begin{description}
\item[{Raise}] \leavevmode
\sphinxcode{\sphinxupquote{psycopg2.errors}}

\end{description}\end{quote}

If the SQL transaction fails, the transaction is rolled back. The most 
recently executed query is printed to \sphinxcode{\sphinxupquote{sys.stderr}}. The error is then 
raised.

\end{fulllineitems}

\index{iterator() (tiger\_leagues.models.db\_model.Database method)@\spxentry{iterator()}\spxextra{tiger\_leagues.models.db\_model.Database method}}

\begin{fulllineitems}
\phantomsection\label{\detokenize{tiger_leagues/models/readme:tiger_leagues.models.db_model.Database.iterator}}\pysiglinewithargsret{\sphinxbfcode{\sphinxupquote{iterator}}}{\emph{cursor}}{}
An alternative to having the \sphinxcode{\sphinxupquote{x = cursor.fetchone(){}` ... 
{}`while x is not None}} dance when iterating through cursor’s results.
\begin{quote}\begin{description}
\item[{Parameters}] \leavevmode
\sphinxstyleliteralstrong{\sphinxupquote{cursor}} \textendash{} psycopg2.cursor

\end{description}\end{quote}

The cursor after after executing the SQL query
\begin{quote}\begin{description}
\item[{Yield}] \leavevmode
\sphinxcode{\sphinxupquote{Row}}

\end{description}\end{quote}

A row fetched from the cursor.
\begin{quote}\begin{description}
\item[{Warn}] \leavevmode
\sphinxcode{\sphinxupquote{DepracationWarning}}

\end{description}\end{quote}

Unlike sqlite3, psycopg2 provides an iterable cursor, so this method 
is unnecessary baggage.

\end{fulllineitems}

\index{launch() (tiger\_leagues.models.db\_model.Database method)@\spxentry{launch()}\spxextra{tiger\_leagues.models.db\_model.Database method}}

\begin{fulllineitems}
\phantomsection\label{\detokenize{tiger_leagues/models/readme:tiger_leagues.models.db_model.Database.launch}}\pysiglinewithargsret{\sphinxbfcode{\sphinxupquote{launch}}}{}{}
Initialize the tables if they do not exist yet.

\end{fulllineitems}


\end{fulllineitems}



\subsection{tiger\_leagues.models.config}
\label{\detokenize{tiger_leagues/models/readme:module-tiger_leagues.models.config}}\label{\detokenize{tiger_leagues/models/readme:tiger-leagues-models-config}}\index{tiger\_leagues.models.config (module)@\spxentry{tiger\_leagues.models.config}\spxextra{module}}
config.py

The central source for variables that span the entire application. As a rule of 
thumb, if you find yourself using \sphinxtitleref{os.environ}, you should probably include the 
variable here instead.

Expected environment variables: \sphinxcode{\sphinxupquote{TIGER\_LEAGUES\_ENVIRONMENT}}, 
\sphinxcode{\sphinxupquote{TIGER\_LEAGUES\_POSTGRESQL\_DBNAME}}, \sphinxcode{\sphinxupquote{TIGER\_LEAGUES\_POSTGRESQL\_USERNAME}}, 
\sphinxcode{\sphinxupquote{TIGER\_LEAGUES\_POSTGRESQL\_PASSWORD}}


\subsection{tiger\_leagues.models.league\_model}
\label{\detokenize{tiger_leagues/models/readme:module-tiger_leagues.models.league_model}}\label{\detokenize{tiger_leagues/models/readme:tiger-leagues-models-league-model}}\index{tiger\_leagues.models.league\_model (module)@\spxentry{tiger\_leagues.models.league\_model}\spxextra{module}}
league.py

Exposes a blueprint that handles requests made to \sphinxtitleref{/league/*} endpoint
\index{create\_league() (in module tiger\_leagues.models.league\_model)@\spxentry{create\_league()}\spxextra{in module tiger\_leagues.models.league\_model}}

\begin{fulllineitems}
\phantomsection\label{\detokenize{tiger_leagues/models/readme:tiger_leagues.models.league_model.create_league}}\pysiglinewithargsret{\sphinxcode{\sphinxupquote{tiger\_leagues.models.league\_model.}}\sphinxbfcode{\sphinxupquote{create\_league}}}{\emph{league\_info}, \emph{creator\_user\_id}}{}~\begin{quote}\begin{description}
\item[{Parameters}] \leavevmode
\sphinxstyleliteralstrong{\sphinxupquote{league\_info}} \textendash{} dict

\end{description}\end{quote}

Expected keys: \sphinxcode{\sphinxupquote{league\_name}}, \sphinxcode{\sphinxupquote{description}}, \sphinxcode{\sphinxupquote{points\_per\_win}}, 
\sphinxcode{\sphinxupquote{points\_per\_draw}}, \sphinxcode{\sphinxupquote{points\_per\_loss}}, \sphinxcode{\sphinxupquote{registration\_deadline}}, 
\sphinxcode{\sphinxupquote{additional\_questions}}.
\begin{quote}\begin{description}
\item[{Parameters}] \leavevmode
\sphinxstyleliteralstrong{\sphinxupquote{creator\_user\_profile}} \textendash{} int

\end{description}\end{quote}

The ID of the user creating this league
\begin{quote}\begin{description}
\item[{Returns}] \leavevmode
\sphinxcode{\sphinxupquote{dict}}

\end{description}\end{quote}

\sphinxcode{\sphinxupquote{success}} is set to \sphinxcode{\sphinxupquote{True}} only if the league was created. 
If \sphinxcode{\sphinxupquote{success}} is \sphinxcode{\sphinxupquote{False}}, the \sphinxcode{\sphinxupquote{message}} field will contain a 
decriptive error message. Otherwise, the \sphinxcode{\sphinxupquote{message}} field will be an 
\sphinxcode{\sphinxupquote{int}} representing the league ID

\end{fulllineitems}

\index{get\_league\_info() (in module tiger\_leagues.models.league\_model)@\spxentry{get\_league\_info()}\spxextra{in module tiger\_leagues.models.league\_model}}

\begin{fulllineitems}
\phantomsection\label{\detokenize{tiger_leagues/models/readme:tiger_leagues.models.league_model.get_league_info}}\pysiglinewithargsret{\sphinxcode{\sphinxupquote{tiger\_leagues.models.league\_model.}}\sphinxbfcode{\sphinxupquote{get\_league\_info}}}{\emph{league\_id}}{}~\begin{quote}\begin{description}
\item[{Parameters}] \leavevmode
\sphinxstyleliteralstrong{\sphinxupquote{league\_id}} \textendash{} int

\end{description}\end{quote}

The ID of this league
\begin{quote}\begin{description}
\item[{Returns}] \leavevmode
\sphinxcode{\sphinxupquote{dict}}

\end{description}\end{quote}

Keys: \sphinxcode{\sphinxupquote{league\_id}}, \sphinxcode{\sphinxupquote{league\_id}}, \sphinxcode{\sphinxupquote{league\_name}}, \sphinxcode{\sphinxupquote{description}}, 
\sphinxcode{\sphinxupquote{points\_per\_win}}, \sphinxcode{\sphinxupquote{points\_per\_draw}}, \sphinxcode{\sphinxupquote{points\_per\_loss}}, \sphinxcode{\sphinxupquote{league\_status}}, 
\sphinxcode{\sphinxupquote{additional\_questions}}, \sphinxcode{\sphinxupquote{registration\_deadline}}, \sphinxcode{\sphinxupquote{num\_games\_per\_period}}, 
\sphinxcode{\sphinxupquote{length\_period\_in\_days}}, \sphinxcode{\sphinxupquote{max\_num\_players}}
\begin{quote}\begin{description}
\item[{Raise}] \leavevmode
\sphinxcode{\sphinxupquote{TigerLeaguesException}}

\end{description}\end{quote}

If the league\_id is not found in the database.

\end{fulllineitems}

\index{get\_league\_info\_if\_joinable() (in module tiger\_leagues.models.league\_model)@\spxentry{get\_league\_info\_if\_joinable()}\spxextra{in module tiger\_leagues.models.league\_model}}

\begin{fulllineitems}
\phantomsection\label{\detokenize{tiger_leagues/models/readme:tiger_leagues.models.league_model.get_league_info_if_joinable}}\pysiglinewithargsret{\sphinxcode{\sphinxupquote{tiger\_leagues.models.league\_model.}}\sphinxbfcode{\sphinxupquote{get\_league\_info\_if\_joinable}}}{\emph{league\_id}}{}~\begin{quote}\begin{description}
\item[{Parameters}] \leavevmode
\sphinxstyleliteralstrong{\sphinxupquote{league\_id}} \textendash{} int

\end{description}\end{quote}

The ID of the league
\begin{quote}\begin{description}
\item[{Returns}] \leavevmode
\sphinxcode{\sphinxupquote{dict}}

\end{description}\end{quote}

If \sphinxcode{\sphinxupquote{succcess}} is \sphinxcode{\sphinxupquote{False}}, \sphinxcode{\sphinxupquote{message}} will contain a descriptive error 
message. If \sphinxcode{\sphinxupquote{success}} is \sphinxcode{\sphinxupquote{True}}, \sphinxcode{\sphinxupquote{message}} will contain the league 
information needed to join the league.

\end{fulllineitems}

\index{get\_league\_standings() (in module tiger\_leagues.models.league\_model)@\spxentry{get\_league\_standings()}\spxextra{in module tiger\_leagues.models.league\_model}}

\begin{fulllineitems}
\phantomsection\label{\detokenize{tiger_leagues/models/readme:tiger_leagues.models.league_model.get_league_standings}}\pysiglinewithargsret{\sphinxcode{\sphinxupquote{tiger\_leagues.models.league\_model.}}\sphinxbfcode{\sphinxupquote{get\_league\_standings}}}{\emph{league\_id}}{}~\begin{quote}\begin{description}
\item[{Parameters}] \leavevmode
\sphinxstyleliteralstrong{\sphinxupquote{league\_id}} \textendash{} int

\end{description}\end{quote}

The ID of the league
\begin{quote}\begin{description}
\item[{Returns}] \leavevmode
\sphinxcode{\sphinxupquote{dict{[}list{[}dict{]}{]}}}

\end{description}\end{quote}

The sorted league standings. The outermost dict is keyed by the division ID. 
The \sphinxcode{\sphinxupquote{list{[}dict{]}}} is sorted by points and then by goal difference. This 
innermost is keyed by \sphinxcode{\sphinxupquote{wins, losses, draws, games\_played, goals\_for, 
goals\_allowed, goal\_diff, points, rank, rank\_delta}}

If the league doesn’t exist, the dict will be empty.

\end{fulllineitems}

\index{get\_leagues\_not\_yet\_joined() (in module tiger\_leagues.models.league\_model)@\spxentry{get\_leagues\_not\_yet\_joined()}\spxextra{in module tiger\_leagues.models.league\_model}}

\begin{fulllineitems}
\phantomsection\label{\detokenize{tiger_leagues/models/readme:tiger_leagues.models.league_model.get_leagues_not_yet_joined}}\pysiglinewithargsret{\sphinxcode{\sphinxupquote{tiger\_leagues.models.league\_model.}}\sphinxbfcode{\sphinxupquote{get\_leagues\_not\_yet\_joined}}}{\emph{user\_profile}}{}~\begin{quote}\begin{description}
\item[{Parameters}] \leavevmode
\sphinxstyleliteralstrong{\sphinxupquote{user\_profile}} \textendash{} dict

\end{description}\end{quote}

Expected keys: \sphinxcode{\sphinxupquote{associated\_leagues}}
\begin{quote}\begin{description}
\item[{Returns}] \leavevmode
\sphinxcode{\sphinxupquote{List{[}DictRow{]}}}

\end{description}\end{quote}

Each item is keyed by \sphinxcode{\sphinxupquote{league\_id}}, \sphinxcode{\sphinxupquote{league\_name}}, \sphinxcode{\sphinxupquote{registration\_deadline}} 
and \sphinxcode{\sphinxupquote{description}}

\end{fulllineitems}

\index{get\_matches\_in\_current\_window() (in module tiger\_leagues.models.league\_model)@\spxentry{get\_matches\_in\_current\_window()}\spxextra{in module tiger\_leagues.models.league\_model}}

\begin{fulllineitems}
\phantomsection\label{\detokenize{tiger_leagues/models/readme:tiger_leagues.models.league_model.get_matches_in_current_window}}\pysiglinewithargsret{\sphinxcode{\sphinxupquote{tiger\_leagues.models.league\_model.}}\sphinxbfcode{\sphinxupquote{get\_matches\_in\_current\_window}}}{\emph{league\_id}, \emph{num\_periods\_before=3}, \emph{num\_periods\_after=3}, \emph{user\_id=None}}{}~\begin{quote}\begin{description}
\item[{Parameters}] \leavevmode
\sphinxstyleliteralstrong{\sphinxupquote{league\_id}} \textendash{} int

\end{description}\end{quote}

The ID of the league
\begin{quote}\begin{description}
\item[{Parameters}] \leavevmode
\sphinxstyleliteralstrong{\sphinxupquote{num\_periods\_before}} \textendash{} int (or infinity)

\end{description}\end{quote}

The fetched matches will have deadlines that are at least on/later than 
\sphinxcode{\sphinxupquote{today - num\_days\_between\_games * num\_periods\_before}}
If the value is \sphinxcode{\sphinxupquote{inf}}, then all matches that have deadlines earlier than 
today will be included in the results.
\begin{quote}\begin{description}
\item[{Parameters}] \leavevmode
\sphinxstyleliteralstrong{\sphinxupquote{num\_periods\_after}} \textendash{} int (or infinity)

\end{description}\end{quote}

The fetched matches will have deadlines that are at least on/earlier than 
\sphinxcode{\sphinxupquote{today + num\_days\_between\_games * num\_periods\_before}}
If the value is \sphinxcode{\sphinxupquote{inf}}, then all matches that have deadlines later than 
today will be included in the results.
\begin{quote}\begin{description}
\item[{Parameters}] \leavevmode
\sphinxstyleliteralstrong{\sphinxupquote{user\_id}} \textendash{} int

\end{description}\end{quote}

If set, only return matches that are associated with this user ID
\begin{quote}\begin{description}
\item[{Returns}] \leavevmode
\sphinxcode{\sphinxupquote{List{[}DictRow{]}}}

\end{description}\end{quote}

A list of all the matches within the current time window. These are the 
matches that are about to be played or have been played. Keys include: 
\sphinxcode{\sphinxupquote{{}`match\_id{}`}} \sphinxcode{\sphinxupquote{{}`user\_1\_id{}`}}, \sphinxcode{\sphinxupquote{{}`user\_2\_id{}`}}, \sphinxcode{\sphinxupquote{{}`league\_id{}`}}, \sphinxcode{\sphinxupquote{{}`division\_id{}`}}, 
\sphinxcode{\sphinxupquote{{}`score\_user\_1{}`}}, \sphinxcode{\sphinxupquote{{}`score\_user\_2{}`}}, \sphinxcode{\sphinxupquote{{}`status{}`}}, \sphinxcode{\sphinxupquote{{}`deadline{}`}}

\end{fulllineitems}

\index{get\_player\_comparison() (in module tiger\_leagues.models.league\_model)@\spxentry{get\_player\_comparison()}\spxextra{in module tiger\_leagues.models.league\_model}}

\begin{fulllineitems}
\phantomsection\label{\detokenize{tiger_leagues/models/readme:tiger_leagues.models.league_model.get_player_comparison}}\pysiglinewithargsret{\sphinxcode{\sphinxupquote{tiger\_leagues.models.league\_model.}}\sphinxbfcode{\sphinxupquote{get\_player\_comparison}}}{\emph{league\_id}, \emph{user\_1\_id}, \emph{user\_2\_id}}{}~\begin{quote}\begin{description}
\item[{Parameters}] \leavevmode
\sphinxstyleliteralstrong{\sphinxupquote{league\_id}} \textendash{} int

\end{description}\end{quote}

The ID of the associated league
\begin{quote}\begin{description}
\item[{Parameters}] \leavevmode
\sphinxstyleliteralstrong{\sphinxupquote{user\_1\_id}} \textendash{} int

\end{description}\end{quote}

The ID of the first user. By convention, user 1 is the initiator of the 
request.
\begin{quote}\begin{description}
\item[{Parameters}] \leavevmode
\sphinxstyleliteralstrong{\sphinxupquote{user\_2\_id}} \textendash{} int

\end{description}\end{quote}

The ID of the second user
\begin{quote}\begin{description}
\item[{Returns}] \leavevmode
\sphinxcode{\sphinxupquote{dict}}

\end{description}\end{quote}

Keyed by \sphinxcode{\sphinxupquote{success}} and \sphinxcode{\sphinxupquote{message}}. 
If \sphinxcode{\sphinxupquote{success}} is \sphinxcode{\sphinxupquote{True}}, \sphinxcode{\sphinxupquote{message}} will be a dict keyed by \sphinxcode{\sphinxupquote{rank, 
points, mutual\_opponents, head\_to\_head, player\_form}}

\end{fulllineitems}

\index{get\_players\_current\_matches() (in module tiger\_leagues.models.league\_model)@\spxentry{get\_players\_current\_matches()}\spxextra{in module tiger\_leagues.models.league\_model}}

\begin{fulllineitems}
\phantomsection\label{\detokenize{tiger_leagues/models/readme:tiger_leagues.models.league_model.get_players_current_matches}}\pysiglinewithargsret{\sphinxcode{\sphinxupquote{tiger\_leagues.models.league\_model.}}\sphinxbfcode{\sphinxupquote{get\_players\_current\_matches}}}{\emph{user\_id}, \emph{league\_id}, \emph{num\_periods\_before=4}, \emph{num\_periods\_after=4}}{}
Unlike \sphinxcode{\sphinxupquote{get\_matches\_in\_current\_window()}}, this method resolves the 
opponent names as well as adding convenient fields such as \sphinxcode{\sphinxupquote{my\_score, 
opponent\_score, opponent\_id, opponent\_name}}.
\begin{quote}\begin{description}
\item[{Parameters}] \leavevmode
\sphinxstyleliteralstrong{\sphinxupquote{user\_id}} \textendash{} int

\end{description}\end{quote}

The ID of the player (equivalent to user)
\begin{quote}\begin{description}
\item[{Parameters}] \leavevmode
\sphinxstyleliteralstrong{\sphinxupquote{league\_id}} \textendash{} int

\end{description}\end{quote}

The ID of the league
\begin{quote}\begin{description}
\item[{Parameters}] \leavevmode
\sphinxstyleliteralstrong{\sphinxupquote{num\_periods\_before}} \textendash{} int (or infinity)

\end{description}\end{quote}

The fetched matches will have deadlines that are at least on/later than 
\sphinxcode{\sphinxupquote{today - num\_days\_between\_games * num\_periods\_before}}
If the value is \sphinxcode{\sphinxupquote{inf}}, then all matches that have deadlines earlier than 
today will be included in the results.
\begin{quote}\begin{description}
\item[{Parameters}] \leavevmode
\sphinxstyleliteralstrong{\sphinxupquote{num\_periods\_after}} \textendash{} int (or infinity)

\end{description}\end{quote}

The fetched matches will have deadlines that are at least on/earlier than 
\sphinxcode{\sphinxupquote{today + num\_days\_between\_games * num\_periods\_before}}
If the value is \sphinxcode{\sphinxupquote{inf}}, then all matches that have deadlines later than 
today will be included in the results.
\begin{quote}\begin{description}
\item[{Returns}] \leavevmode
\sphinxcode{\sphinxupquote{List{[}dict{]}}}

\end{description}\end{quote}

The player’s current matches ordered by the deadline. Each dict is keyed by 
\sphinxcode{\sphinxupquote{match\_id, league\_id, division\_id, status, deadline, my\_score,  
opponent\_id, opponent\_score, opponent\_name}}

\end{fulllineitems}

\index{get\_players\_league\_stats() (in module tiger\_leagues.models.league\_model)@\spxentry{get\_players\_league\_stats()}\spxextra{in module tiger\_leagues.models.league\_model}}

\begin{fulllineitems}
\phantomsection\label{\detokenize{tiger_leagues/models/readme:tiger_leagues.models.league_model.get_players_league_stats}}\pysiglinewithargsret{\sphinxcode{\sphinxupquote{tiger\_leagues.models.league\_model.}}\sphinxbfcode{\sphinxupquote{get\_players\_league\_stats}}}{\emph{league\_id}, \emph{user\_id}, \emph{matches=None}, \emph{k=5}}{}~\begin{quote}\begin{description}
\item[{Parameters}] \leavevmode
\sphinxstyleliteralstrong{\sphinxupquote{league\_id}} \textendash{} int

\end{description}\end{quote}

The ID of the associated league
\begin{quote}\begin{description}
\item[{Parameters}] \leavevmode
\sphinxstyleliteralstrong{\sphinxupquote{user\_id}} \textendash{} int

\end{description}\end{quote}

The ID of the user
\begin{quote}\begin{description}
\item[{Parameters}] \leavevmode
\sphinxstyleliteralstrong{\sphinxupquote{matches}} \textendash{} \sphinxcode{\sphinxupquote{list{[}dict{]}}}

\end{description}\end{quote}

A list of matches to calculate the player’s form from. The matches should be 
ordered such that the most recent matches appear last in the list.
If \sphinxcode{\sphinxupquote{NoneType}}, this method queries the database for such matches.
\begin{quote}\begin{description}
\item[{Parameters}] \leavevmode
\sphinxstyleliteralstrong{\sphinxupquote{k}} \textendash{} int

\end{description}\end{quote}

The max number of matches to calculate a player’s form from.
\begin{quote}\begin{description}
\item[{Returns}] \leavevmode
\sphinxcode{\sphinxupquote{dict}}

\end{description}\end{quote}

Keyed by \sphinxcode{\sphinxupquote{success}} and \sphinxcode{\sphinxupquote{message}}. 
If \sphinxcode{\sphinxupquote{success}} is \sphinxcode{\sphinxupquote{True}}, \sphinxcode{\sphinxupquote{message}} will be a dict keyed by \sphinxcode{\sphinxupquote{rank, 
points, player\_form}}
If \sphinxcode{\sphinxupquote{success}} is \sphinxcode{\sphinxupquote{False}}, \sphinxcode{\sphinxupquote{message}} will contain a descriptive error 
message.

\end{fulllineitems}

\index{get\_previous\_responses() (in module tiger\_leagues.models.league\_model)@\spxentry{get\_previous\_responses()}\spxextra{in module tiger\_leagues.models.league\_model}}

\begin{fulllineitems}
\phantomsection\label{\detokenize{tiger_leagues/models/readme:tiger_leagues.models.league_model.get_previous_responses}}\pysiglinewithargsret{\sphinxcode{\sphinxupquote{tiger\_leagues.models.league\_model.}}\sphinxbfcode{\sphinxupquote{get\_previous\_responses}}}{\emph{league\_id}, \emph{user\_profile}}{}~\begin{quote}\begin{description}
\item[{Parameters}] \leavevmode
\sphinxstyleliteralstrong{\sphinxupquote{league\_id}} \textendash{} int

\end{description}\end{quote}

The ID of the league
\begin{quote}\begin{description}
\item[{Parameters}] \leavevmode
\sphinxstyleliteralstrong{\sphinxupquote{user\_profile}} \textendash{} dict

\end{description}\end{quote}

Expected keys: \sphinxcode{\sphinxupquote{associated\_leagues, user\_id}}
\begin{quote}\begin{description}
\item[{Returns}] \leavevmode
\sphinxcode{\sphinxupquote{NoneType}}

\end{description}\end{quote}

If the user has not tried to join this league before
\begin{quote}\begin{description}
\item[{Returns}] \leavevmode
\sphinxcode{\sphinxupquote{DictRow}}

\end{description}\end{quote}

The responses that the user previously entered while trying to join this 
league.

\end{fulllineitems}

\index{process\_join\_league\_request() (in module tiger\_leagues.models.league\_model)@\spxentry{process\_join\_league\_request()}\spxextra{in module tiger\_leagues.models.league\_model}}

\begin{fulllineitems}
\phantomsection\label{\detokenize{tiger_leagues/models/readme:tiger_leagues.models.league_model.process_join_league_request}}\pysiglinewithargsret{\sphinxcode{\sphinxupquote{tiger\_leagues.models.league\_model.}}\sphinxbfcode{\sphinxupquote{process\_join\_league\_request}}}{\emph{league\_id}, \emph{user\_profile}, \emph{submitted\_data}}{}~\begin{quote}\begin{description}
\item[{Parameters}] \leavevmode
\sphinxstyleliteralstrong{\sphinxupquote{league\_id}} \textendash{} int

\end{description}\end{quote}

The ID of this league
\begin{quote}\begin{description}
\item[{Parameters}] \leavevmode
\sphinxstyleliteralstrong{\sphinxupquote{user\_profile}} \textendash{} dict

\end{description}\end{quote}

Expected keys: \sphinxcode{\sphinxupquote{user\_id, league\_ids}}
\begin{quote}\begin{description}
\item[{Returns}] \leavevmode
\sphinxcode{\sphinxupquote{dict}}

\end{description}\end{quote}

If \sphinxcode{\sphinxupquote{succcess}} is \sphinxcode{\sphinxupquote{False}}, \sphinxcode{\sphinxupquote{message}} will contain a descriptive error 
message. If \sphinxcode{\sphinxupquote{success}} is \sphinxcode{\sphinxupquote{True}}, \sphinxcode{\sphinxupquote{message}} will contain a 
dict containing the updated user profile.

\end{fulllineitems}

\index{process\_leave\_league\_request() (in module tiger\_leagues.models.league\_model)@\spxentry{process\_leave\_league\_request()}\spxextra{in module tiger\_leagues.models.league\_model}}

\begin{fulllineitems}
\phantomsection\label{\detokenize{tiger_leagues/models/readme:tiger_leagues.models.league_model.process_leave_league_request}}\pysiglinewithargsret{\sphinxcode{\sphinxupquote{tiger\_leagues.models.league\_model.}}\sphinxbfcode{\sphinxupquote{process\_leave\_league\_request}}}{\emph{league\_id}, \emph{user\_profile}}{}~\begin{quote}\begin{description}
\item[{Parameters}] \leavevmode
\sphinxstyleliteralstrong{\sphinxupquote{league\_id}} \textendash{} int

\end{description}\end{quote}

The ID of the league
\begin{quote}\begin{description}
\item[{Parameters}] \leavevmode
\sphinxstyleliteralstrong{\sphinxupquote{user\_profile}} \textendash{} dict

\end{description}\end{quote}

Expected keys: \sphinxcode{\sphinxupquote{user\_id, league\_ids}}
\begin{quote}\begin{description}
\item[{Returns}] \leavevmode
\sphinxcode{\sphinxupquote{bool}}:

\end{description}\end{quote}

\sphinxcode{\sphinxupquote{True}} if the user was successfully removed from the league, \sphinxcode{\sphinxupquote{False}} 
otherwise

\end{fulllineitems}

\index{process\_player\_score\_report() (in module tiger\_leagues.models.league\_model)@\spxentry{process\_player\_score\_report()}\spxextra{in module tiger\_leagues.models.league\_model}}

\begin{fulllineitems}
\phantomsection\label{\detokenize{tiger_leagues/models/readme:tiger_leagues.models.league_model.process_player_score_report}}\pysiglinewithargsret{\sphinxcode{\sphinxupquote{tiger\_leagues.models.league\_model.}}\sphinxbfcode{\sphinxupquote{process\_player\_score\_report}}}{\emph{user\_id}, \emph{score\_details}}{}~\begin{quote}\begin{description}
\item[{Parameters}] \leavevmode
\sphinxstyleliteralstrong{\sphinxupquote{user\_id}} \textendash{} int

\end{description}\end{quote}

The ID of the user submitting the score report
\begin{quote}\begin{description}
\item[{Parameters}] \leavevmode
\sphinxstyleliteralstrong{\sphinxupquote{score\_details}} \textendash{} dict

\end{description}\end{quote}

Expected keys: \sphinxcode{\sphinxupquote{my\_score}}, \sphinxcode{\sphinxupquote{opponent\_score}}, \sphinxcode{\sphinxupquote{match\_id}}.
\begin{quote}\begin{description}
\item[{Returns}] \leavevmode
\sphinxcode{\sphinxupquote{{}`dict{}`}}

\end{description}\end{quote}

Keys: \sphinxcode{\sphinxupquote{{}`success{}`}}, \sphinxcode{\sphinxupquote{{}`message{}`}}. If \sphinxcode{\sphinxupquote{{}`success{}`}} is \sphinxcode{\sphinxupquote{{}`True{}`}}, \sphinxcode{\sphinxupquote{{}`message{}`}} 
contains the status of the match after the score has been processed. 
Otherwise, \sphinxcode{\sphinxupquote{{}`message{}`}} contains an explanation of what went wrong.

\end{fulllineitems}

\index{process\_update\_league\_responses() (in module tiger\_leagues.models.league\_model)@\spxentry{process\_update\_league\_responses()}\spxextra{in module tiger\_leagues.models.league\_model}}

\begin{fulllineitems}
\phantomsection\label{\detokenize{tiger_leagues/models/readme:tiger_leagues.models.league_model.process_update_league_responses}}\pysiglinewithargsret{\sphinxcode{\sphinxupquote{tiger\_leagues.models.league\_model.}}\sphinxbfcode{\sphinxupquote{process\_update\_league\_responses}}}{\emph{league\_id}, \emph{user\_profile}, \emph{submitted\_data}}{}~\begin{quote}\begin{description}
\item[{Parameters}] \leavevmode
\sphinxstyleliteralstrong{\sphinxupquote{league\_id}} \textendash{} int

\end{description}\end{quote}

The ID of this league
\begin{quote}\begin{description}
\item[{Parameters}] \leavevmode
\sphinxstyleliteralstrong{\sphinxupquote{user\_profile}} \textendash{} dict

\end{description}\end{quote}

Expected keys: \sphinxcode{\sphinxupquote{user\_id, league\_ids}}
\begin{quote}\begin{description}
\item[{Returns}] \leavevmode
\sphinxcode{\sphinxupquote{dict}}

\end{description}\end{quote}

If \sphinxcode{\sphinxupquote{succcess}} is \sphinxcode{\sphinxupquote{False}}, \sphinxcode{\sphinxupquote{message}} will contain a descriptive error 
message. If \sphinxcode{\sphinxupquote{success}} is \sphinxcode{\sphinxupquote{True}}, \sphinxcode{\sphinxupquote{message}} will contain a 
dict containing the updated user profile.

\end{fulllineitems}

\index{update\_league\_standings() (in module tiger\_leagues.models.league\_model)@\spxentry{update\_league\_standings()}\spxextra{in module tiger\_leagues.models.league\_model}}

\begin{fulllineitems}
\phantomsection\label{\detokenize{tiger_leagues/models/readme:tiger_leagues.models.league_model.update_league_standings}}\pysiglinewithargsret{\sphinxcode{\sphinxupquote{tiger\_leagues.models.league\_model.}}\sphinxbfcode{\sphinxupquote{update\_league\_standings}}}{\emph{league\_id}, \emph{division\_id}}{}
Compute the new league standings and persist them into the database.
\begin{quote}\begin{description}
\item[{Parameters}] \leavevmode
\sphinxstyleliteralstrong{\sphinxupquote{league\_id}} \textendash{} int

\end{description}\end{quote}

The ID of the league
\begin{quote}\begin{description}
\item[{Parameters}] \leavevmode
\sphinxstyleliteralstrong{\sphinxupquote{division\_id}} \textendash{} int

\end{description}\end{quote}

The ID of the division within the league of interest
\begin{quote}\begin{description}
\item[{Returns}] \leavevmode
\sphinxcode{\sphinxupquote{NoneType}}

\end{description}\end{quote}

This method affects the state of the database. It doesn’t return anything. 
To fetch the standings, call \sphinxcode{\sphinxupquote{get\_league\_standings()}} instead.

\end{fulllineitems}



\subsection{tiger\_leagues.models.admin\_model}
\label{\detokenize{tiger_leagues/models/readme:module-tiger_leagues.models.admin_model}}\label{\detokenize{tiger_leagues/models/readme:tiger-leagues-models-admin-model}}\index{tiger\_leagues.models.admin\_model (module)@\spxentry{tiger\_leagues.models.admin\_model}\spxextra{module}}
admin\_model.py

Exposes functions that are used by the controller for the \sphinxtitleref{/admin/*} endpoint
\index{allocate\_league\_divisions() (in module tiger\_leagues.models.admin\_model)@\spxentry{allocate\_league\_divisions()}\spxextra{in module tiger\_leagues.models.admin\_model}}

\begin{fulllineitems}
\phantomsection\label{\detokenize{tiger_leagues/models/readme:tiger_leagues.models.admin_model.allocate_league_divisions}}\pysiglinewithargsret{\sphinxcode{\sphinxupquote{tiger\_leagues.models.admin\_model.}}\sphinxbfcode{\sphinxupquote{allocate\_league\_divisions}}}{\emph{league\_id}, \emph{desired\_allocation\_config}}{}~\begin{quote}\begin{description}
\item[{Parameters}] \leavevmode
\sphinxstyleliteralstrong{\sphinxupquote{league\_id}} \textendash{} int

\end{description}\end{quote}

The ID of the league
\begin{quote}\begin{description}
\item[{Parameters}] \leavevmode
\sphinxstyleliteralstrong{\sphinxupquote{desired\_allocation\_config}} \textendash{} dict

\end{description}\end{quote}

Options to use when allocating the divisions. Keys may include 
\sphinxcode{\sphinxupquote{num\_games\_per\_period}}, \sphinxcode{\sphinxupquote{length\_period\_in\_days}}, \sphinxcode{\sphinxupquote{completion\_deadline}}
\begin{quote}\begin{description}
\item[{Returns}] \leavevmode
\sphinxcode{\sphinxupquote{dict}}

\end{description}\end{quote}

If \sphinxcode{\sphinxupquote{success}} is \sphinxcode{\sphinxupquote{False}}, \sphinxcode{\sphinxupquote{message}} contains a string describing what 
went wrong. 
Otherwise, \sphinxcode{\sphinxupquote{message}} is a dict keyed by \sphinxcode{\sphinxupquote{divisions}} and \sphinxcode{\sphinxupquote{end\_date}}

\end{fulllineitems}

\index{approve\_match() (in module tiger\_leagues.models.admin\_model)@\spxentry{approve\_match()}\spxextra{in module tiger\_leagues.models.admin\_model}}

\begin{fulllineitems}
\phantomsection\label{\detokenize{tiger_leagues/models/readme:tiger_leagues.models.admin_model.approve_match}}\pysiglinewithargsret{\sphinxcode{\sphinxupquote{tiger\_leagues.models.admin\_model.}}\sphinxbfcode{\sphinxupquote{approve\_match}}}{\emph{score\_info}, \emph{admin\_user\_id}}{}~\begin{quote}\begin{description}
\item[{Parameters}] \leavevmode
\sphinxstyleliteralstrong{\sphinxupquote{score\_info}} \textendash{} dict

\end{description}\end{quote}

Expected keys: \sphinxcode{\sphinxupquote{score\_user\_1}}, \sphinxcode{\sphinxupquote{score\_user\_2}}, \sphinxcode{\sphinxupquote{match\_id}}
\begin{quote}\begin{description}
\item[{Parameters}] \leavevmode
\sphinxstyleliteralstrong{\sphinxupquote{admin\_user\_id}} \textendash{} int

\end{description}\end{quote}

The ID of the admin approving the match’s results
\begin{quote}\begin{description}
\item[{Returns}] \leavevmode
\sphinxcode{\sphinxupquote{dict}}

\end{description}\end{quote}

Keys: \sphinxcode{\sphinxupquote{success}}, \sphinxcode{\sphinxupquote{message}}. If \sphinxcode{\sphinxupquote{success}} is \sphinxcode{\sphinxupquote{True}}, \sphinxcode{\sphinxupquote{message}} has 
the updated status of the match. Otherwise, \sphinxcode{\sphinxupquote{message}} explains why the 
update failed.

\end{fulllineitems}

\index{delete\_league() (in module tiger\_leagues.models.admin\_model)@\spxentry{delete\_league()}\spxextra{in module tiger\_leagues.models.admin\_model}}

\begin{fulllineitems}
\phantomsection\label{\detokenize{tiger_leagues/models/readme:tiger_leagues.models.admin_model.delete_league}}\pysiglinewithargsret{\sphinxcode{\sphinxupquote{tiger\_leagues.models.admin\_model.}}\sphinxbfcode{\sphinxupquote{delete\_league}}}{\emph{league\_id}}{}~\begin{quote}\begin{description}
\item[{Parameters}] \leavevmode
\sphinxstyleliteralstrong{\sphinxupquote{league\_id}} \textendash{} int

\end{description}\end{quote}

The ID of the league
\begin{quote}\begin{description}
\item[{Returns}] \leavevmode
\sphinxcode{\sphinxupquote{dict}}

\end{description}\end{quote}

Keys: \sphinxcode{\sphinxupquote{success}}, \sphinxcode{\sphinxupquote{message}}. If \sphinxcode{\sphinxupquote{success}} is \sphinxcode{\sphinxupquote{True}}, \sphinxcode{\sphinxupquote{message}} has 
a confirmation message. Otherwise, \sphinxcode{\sphinxupquote{message}} explains why the deletion failed.

\end{fulllineitems}

\index{fixture\_generator() (in module tiger\_leagues.models.admin\_model)@\spxentry{fixture\_generator()}\spxextra{in module tiger\_leagues.models.admin\_model}}

\begin{fulllineitems}
\phantomsection\label{\detokenize{tiger_leagues/models/readme:tiger_leagues.models.admin_model.fixture_generator}}\pysiglinewithargsret{\sphinxcode{\sphinxupquote{tiger\_leagues.models.admin\_model.}}\sphinxbfcode{\sphinxupquote{fixture\_generator}}}{\emph{user\_ids}}{}~\begin{quote}\begin{description}
\item[{Parameters}] \leavevmode
\sphinxstyleliteralstrong{\sphinxupquote{user\_ids}} \textendash{} List{[}int{]}

\end{description}\end{quote}

A list of the IDs of users who are supposed to play each other.
\begin{quote}\begin{description}
\item[{Returns}] \leavevmode
\sphinxcode{\sphinxupquote{List{[}List{[}List{]}{]}}}

\end{description}\end{quote}

The innermost list has 2 elements (the IDs of the players involved in a game). 
The middle list has a collection of all the games being played at a 
particular timeslot. The outermost list encompasses all the games that will 
be played between all the users.

\end{fulllineitems}

\index{generate\_league\_fixtures() (in module tiger\_leagues.models.admin\_model)@\spxentry{generate\_league\_fixtures()}\spxextra{in module tiger\_leagues.models.admin\_model}}

\begin{fulllineitems}
\phantomsection\label{\detokenize{tiger_leagues/models/readme:tiger_leagues.models.admin_model.generate_league_fixtures}}\pysiglinewithargsret{\sphinxcode{\sphinxupquote{tiger\_leagues.models.admin\_model.}}\sphinxbfcode{\sphinxupquote{generate\_league\_fixtures}}}{\emph{league\_id}, \emph{div\_allocations}, \emph{start\_date=None}}{}~\begin{quote}\begin{description}
\item[{Parameters}] \leavevmode
\sphinxstyleliteralstrong{\sphinxupquote{league\_id}} \textendash{} int

\end{description}\end{quote}

The ID of the league
\begin{quote}\begin{description}
\item[{Parameters}] \leavevmode
\sphinxstyleliteralstrong{\sphinxupquote{div\_allocations}} \textendash{} dict

\end{description}\end{quote}

The keys are the division IDs. Each value is a dict keyed by \sphinxcode{\sphinxupquote{name}} and 
\sphinxcode{\sphinxupquote{user\_id}} representing a player associated with the league.
\begin{quote}\begin{description}
\item[{Parameters}] \leavevmode
\sphinxstyleliteralstrong{\sphinxupquote{start\_date}} \textendash{} \sphinxcode{\sphinxupquote{date}}

\end{description}\end{quote}

The earliest games’ time window will start from this date. Defaults to 
tomorrow
\begin{quote}\begin{description}
\item[{Returns}] \leavevmode
\sphinxcode{\sphinxupquote{dict}}

\end{description}\end{quote}

If \sphinxcode{\sphinxupquote{success}} is \sphinxcode{\sphinxupquote{False}}, \sphinxcode{\sphinxupquote{message}} will have a description of why the 
call failed. Otherwise, \sphinxcode{\sphinxupquote{message}} will contain a string confirming that 
the fixtures were generated.

\end{fulllineitems}

\index{get\_current\_matches() (in module tiger\_leagues.models.admin\_model)@\spxentry{get\_current\_matches()}\spxextra{in module tiger\_leagues.models.admin\_model}}

\begin{fulllineitems}
\phantomsection\label{\detokenize{tiger_leagues/models/readme:tiger_leagues.models.admin_model.get_current_matches}}\pysiglinewithargsret{\sphinxcode{\sphinxupquote{tiger\_leagues.models.admin\_model.}}\sphinxbfcode{\sphinxupquote{get\_current\_matches}}}{\emph{league\_id}}{}~\begin{quote}\begin{description}
\item[{Parameters}] \leavevmode
\sphinxstyleliteralstrong{\sphinxupquote{league\_id}} \textendash{} int

\end{description}\end{quote}

The ID of the league
\begin{quote}\begin{description}
\item[{Returns}] \leavevmode
\sphinxcode{\sphinxupquote{List{[}DictRow{]}}}

\end{description}\end{quote}

A list of all matches in the current time block. 
Keys include \sphinxcode{\sphinxupquote{match\_id}}, \sphinxcode{\sphinxupquote{league\_id}}, \sphinxcode{\sphinxupquote{user\_1\_id}}, \sphinxcode{\sphinxupquote{user\_2\_id}}, 
\sphinxcode{\sphinxupquote{division\_id}}, \sphinxcode{\sphinxupquote{score\_user\_1}}, \sphinxcode{\sphinxupquote{score\_user\_2}}, \sphinxcode{\sphinxupquote{status}}, 
\sphinxcode{\sphinxupquote{user\_1\_name}}, \sphinxcode{\sphinxupquote{user\_2\_name}}

\end{fulllineitems}

\index{get\_join\_league\_requests() (in module tiger\_leagues.models.admin\_model)@\spxentry{get\_join\_league\_requests()}\spxextra{in module tiger\_leagues.models.admin\_model}}

\begin{fulllineitems}
\phantomsection\label{\detokenize{tiger_leagues/models/readme:tiger_leagues.models.admin_model.get_join_league_requests}}\pysiglinewithargsret{\sphinxcode{\sphinxupquote{tiger\_leagues.models.admin\_model.}}\sphinxbfcode{\sphinxupquote{get\_join\_league\_requests}}}{\emph{league\_id}}{}~\begin{quote}\begin{description}
\item[{Parameters}] \leavevmode
\sphinxstyleliteralstrong{\sphinxupquote{league\_id}} \textendash{} int

\end{description}\end{quote}

The ID of the league
\begin{quote}\begin{description}
\item[{Returns}] \leavevmode
\sphinxcode{\sphinxupquote{List{[}DictCursor{]}}}

\end{description}\end{quote}

A row for each user who submitted a request to join this league.

\end{fulllineitems}

\index{get\_registration\_stats() (in module tiger\_leagues.models.admin\_model)@\spxentry{get\_registration\_stats()}\spxextra{in module tiger\_leagues.models.admin\_model}}

\begin{fulllineitems}
\phantomsection\label{\detokenize{tiger_leagues/models/readme:tiger_leagues.models.admin_model.get_registration_stats}}\pysiglinewithargsret{\sphinxcode{\sphinxupquote{tiger\_leagues.models.admin\_model.}}\sphinxbfcode{\sphinxupquote{get\_registration\_stats}}}{\emph{league\_id}}{}~\begin{quote}\begin{description}
\item[{Parameters}] \leavevmode
\sphinxstyleliteralstrong{\sphinxupquote{league\_id}} \textendash{} int

\end{description}\end{quote}

The ID of the league
\begin{quote}\begin{description}
\item[{Returns}] \leavevmode
\sphinxcode{\sphinxupquote{dict}}

\end{description}\end{quote}

The keys are various join statuses and the values are their frequency.

\end{fulllineitems}

\index{update\_join\_league\_requests() (in module tiger\_leagues.models.admin\_model)@\spxentry{update\_join\_league\_requests()}\spxextra{in module tiger\_leagues.models.admin\_model}}

\begin{fulllineitems}
\phantomsection\label{\detokenize{tiger_leagues/models/readme:tiger_leagues.models.admin_model.update_join_league_requests}}\pysiglinewithargsret{\sphinxcode{\sphinxupquote{tiger\_leagues.models.admin\_model.}}\sphinxbfcode{\sphinxupquote{update\_join\_league\_requests}}}{\emph{league\_id}, \emph{league\_statuses}}{}~\begin{quote}\begin{description}
\item[{Parameters}] \leavevmode
\sphinxstyleliteralstrong{\sphinxupquote{league\_id}} \textendash{} int

\end{description}\end{quote}

The ID of the league
\begin{quote}\begin{description}
\item[{Parameters}] \leavevmode
\sphinxstyleliteralstrong{\sphinxupquote{league\_statuses}} \textendash{} dict

\end{description}\end{quote}

The keys are user\_ids and the values are any of the supported status strings
\begin{quote}\begin{description}
\item[{Returns}] \leavevmode
\sphinxcode{\sphinxupquote{dict}}

\end{description}\end{quote}

If \sphinxcode{\sphinxupquote{success}} is set, \sphinxcode{\sphinxupquote{message}} will contain a user\_id-\textgreater{}status matching. 
Otherwise, \sphinxcode{\sphinxupquote{message}} will contain an error description.

\end{fulllineitems}



\subsection{tiger\_leagues.models.user\_model}
\label{\detokenize{tiger_leagues/models/readme:module-tiger_leagues.models.user_model}}\label{\detokenize{tiger_leagues/models/readme:tiger-leagues-models-user-model}}\index{tiger\_leagues.models.user\_model (module)@\spxentry{tiger\_leagues.models.user\_model}\spxextra{module}}
user\_model.py

Exposes functions that are used by the controller for the \sphinxtitleref{/user/*} endpoint
\index{get\_user() (in module tiger\_leagues.models.user\_model)@\spxentry{get\_user()}\spxextra{in module tiger\_leagues.models.user\_model}}

\begin{fulllineitems}
\phantomsection\label{\detokenize{tiger_leagues/models/readme:tiger_leagues.models.user_model.get_user}}\pysiglinewithargsret{\sphinxcode{\sphinxupquote{tiger\_leagues.models.user\_model.}}\sphinxbfcode{\sphinxupquote{get\_user}}}{\emph{net\_id}, \emph{user\_id=None}}{}~\begin{quote}\begin{description}
\item[{Parameters}] \leavevmode
\sphinxstyleliteralstrong{\sphinxupquote{net\_id}} \textendash{} str

\end{description}\end{quote}

The Princeton Net ID of the user
\begin{quote}\begin{description}
\item[{Parameters}] \leavevmode
\sphinxstyleliteralstrong{\sphinxupquote{user\_id}} \textendash{} int

\end{description}\end{quote}

The ID of the user as assigned in Tiger Leagues
\begin{quote}\begin{description}
\item[{Returns}] \leavevmode
\sphinxcode{\sphinxupquote{dict}}

\end{description}\end{quote}

A representation of the user as stored in the database. Keys include: 
\sphinxcode{\sphinxupquote{user\_id, name, net\_id, email, phone\_num, room, league\_ids, 
associated\_leagues, unread\_notifications}}
\begin{quote}\begin{description}
\item[{Returns}] \leavevmode
\sphinxcode{\sphinxupquote{NoneType}}

\end{description}\end{quote}

If there is no user in the database with the provided net id

\end{fulllineitems}

\index{read\_notifications() (in module tiger\_leagues.models.user\_model)@\spxentry{read\_notifications()}\spxextra{in module tiger\_leagues.models.user\_model}}

\begin{fulllineitems}
\phantomsection\label{\detokenize{tiger_leagues/models/readme:tiger_leagues.models.user_model.read_notifications}}\pysiglinewithargsret{\sphinxcode{\sphinxupquote{tiger\_leagues.models.user\_model.}}\sphinxbfcode{\sphinxupquote{read\_notifications}}}{\emph{user\_id}, \emph{notification\_status=None}}{}~\begin{quote}\begin{description}
\item[{Parameters}] \leavevmode
\sphinxstyleliteralstrong{\sphinxupquote{user\_id}} \textendash{} int

\end{description}\end{quote}

The ID of the associated user
\begin{quote}\begin{description}
\item[{Parameters}] \leavevmode
\sphinxstyleliteralstrong{\sphinxupquote{notification\_status}} \textendash{} str

\end{description}\end{quote}

The status of the notifications that are to be read. If \sphinxcode{\sphinxupquote{None}}, this defaults 
to notifications that have not been archived.
\begin{quote}\begin{description}
\item[{Returns}] \leavevmode
\sphinxcode{\sphinxupquote{cursor}}

\end{description}\end{quote}

An iterable cursor where each item keyed by \sphinxcode{\sphinxupquote{notification\_id}}, 
\sphinxcode{\sphinxupquote{notification\_status}}, \sphinxcode{\sphinxupquote{notification\_text}}, \sphinxcode{\sphinxupquote{created\_at}}, \sphinxcode{\sphinxupquote{league\_name}}.

\end{fulllineitems}

\index{send\_notification() (in module tiger\_leagues.models.user\_model)@\spxentry{send\_notification()}\spxextra{in module tiger\_leagues.models.user\_model}}

\begin{fulllineitems}
\phantomsection\label{\detokenize{tiger_leagues/models/readme:tiger_leagues.models.user_model.send_notification}}\pysiglinewithargsret{\sphinxcode{\sphinxupquote{tiger\_leagues.models.user\_model.}}\sphinxbfcode{\sphinxupquote{send\_notification}}}{\emph{user\_id}, \emph{notification}}{}
Send a notification to this user
\begin{quote}\begin{description}
\item[{Parameters}] \leavevmode
\sphinxstyleliteralstrong{\sphinxupquote{user\_id}} \textendash{} int

\end{description}\end{quote}

The ID of the associated user
\begin{quote}\begin{description}
\item[{Parameters}] \leavevmode
\sphinxstyleliteralstrong{\sphinxupquote{notification}} \textendash{} dict

\end{description}\end{quote}

Expected keys include: \sphinxcode{\sphinxupquote{league\_id, notification\_text}}
\begin{quote}\begin{description}
\item[{Returns}] \leavevmode
\sphinxcode{\sphinxupquote{int}}

\end{description}\end{quote}

The notification ID if the notification is successfully delivered to the user’s 
mailbox.
\begin{quote}\begin{description}
\item[{Returns}] \leavevmode
\sphinxcode{\sphinxupquote{NoneType}}

\end{description}\end{quote}

If the method failed

\end{fulllineitems}

\index{update\_notification\_status() (in module tiger\_leagues.models.user\_model)@\spxentry{update\_notification\_status()}\spxextra{in module tiger\_leagues.models.user\_model}}

\begin{fulllineitems}
\phantomsection\label{\detokenize{tiger_leagues/models/readme:tiger_leagues.models.user_model.update_notification_status}}\pysiglinewithargsret{\sphinxcode{\sphinxupquote{tiger\_leagues.models.user\_model.}}\sphinxbfcode{\sphinxupquote{update\_notification\_status}}}{\emph{user\_id}, \emph{notification\_obj}}{}~\begin{quote}\begin{description}
\item[{Parameters}] \leavevmode
\sphinxstyleliteralstrong{\sphinxupquote{user\_id}} \textendash{} int

\end{description}\end{quote}

The ID of the user making this request
\begin{quote}\begin{description}
\item[{Parameters}] \leavevmode
\sphinxstyleliteralstrong{\sphinxupquote{notification\_obj}} \textendash{} dict

\end{description}\end{quote}

Expected keys: \sphinxcode{\sphinxupquote{notification\_id}}, \sphinxcode{\sphinxupquote{notification\_status}}
\begin{quote}\begin{description}
\item[{Returns}] \leavevmode
\sphinxcode{\sphinxupquote{dict}}

\end{description}\end{quote}

Keyed by \sphinxcode{\sphinxupquote{success}} and \sphinxcode{\sphinxupquote{message}}. 
If \sphinxcode{\sphinxupquote{success}} is \sphinxcode{\sphinxupquote{False}}, \sphinxcode{\sphinxupquote{message}} contains a description of why the 
request failed.
If \sphinxcode{\sphinxupquote{success}} is \sphinxcode{\sphinxupquote{True}}, \sphinxcode{\sphinxupquote{message}} contains the new status of the 
notification.

\end{fulllineitems}

\index{update\_user\_profile() (in module tiger\_leagues.models.user\_model)@\spxentry{update\_user\_profile()}\spxextra{in module tiger\_leagues.models.user\_model}}

\begin{fulllineitems}
\phantomsection\label{\detokenize{tiger_leagues/models/readme:tiger_leagues.models.user_model.update_user_profile}}\pysiglinewithargsret{\sphinxcode{\sphinxupquote{tiger\_leagues.models.user\_model.}}\sphinxbfcode{\sphinxupquote{update\_user\_profile}}}{\emph{user\_profile}, \emph{net\_id}, \emph{submitted\_data}}{}~\begin{quote}\begin{description}
\item[{Parameters}] \leavevmode
\sphinxstyleliteralstrong{\sphinxupquote{user\_profile}} \textendash{} dict

\end{description}\end{quote}

A representation of the user, usually obtained from \sphinxcode{\sphinxupquote{get\_user(net\_id)}}. If 
set to \sphinxcode{\sphinxupquote{None}}, a new user will be created and added to the database.
\begin{quote}\begin{description}
\item[{Parameters}] \leavevmode
\sphinxstyleliteralstrong{\sphinxupquote{net\_id}} \textendash{} str

\end{description}\end{quote}

The Princeton Net ID of the user
\begin{quote}\begin{description}
\item[{Parameters}] \leavevmode
\sphinxstyleliteralstrong{\sphinxupquote{submitted\_data}} \textendash{} dict

\end{description}\end{quote}

Keys may include \sphinxtitleref{name}, \sphinxtitleref{email}, \sphinxtitleref{phone\_num}, \sphinxtitleref{room}
\begin{quote}\begin{description}
\item[{Returns}] \leavevmode
dict

\end{description}\end{quote}

The updated user profile

\end{fulllineitems}



\subsection{tiger\_leagues.models.exception}
\label{\detokenize{tiger_leagues/models/readme:module-tiger_leagues.models.exception}}\label{\detokenize{tiger_leagues/models/readme:tiger-leagues-models-exception}}\index{tiger\_leagues.models.exception (module)@\spxentry{tiger\_leagues.models.exception}\spxextra{module}}
exception.py

Allows for error pages/responses with custom exception messages.
\index{TigerLeaguesException@\spxentry{TigerLeaguesException}}

\begin{fulllineitems}
\phantomsection\label{\detokenize{tiger_leagues/models/readme:tiger_leagues.models.exception.TigerLeaguesException}}\pysiglinewithargsret{\sphinxbfcode{\sphinxupquote{exception }}\sphinxcode{\sphinxupquote{tiger\_leagues.models.exception.}}\sphinxbfcode{\sphinxupquote{TigerLeaguesException}}}{\emph{message}, \emph{status\_code=400}, \emph{jsonify=True}}{}
A special exception for errors that arise due to constraints that we set on 
the application, for instance, a user may not access the league panel for a 
league in which they lack an admin status, etc.
\begin{quote}\begin{description}
\item[{Parameters}] \leavevmode
\sphinxstyleliteralstrong{\sphinxupquote{message}} \textendash{} str

\end{description}\end{quote}

human readable string explaining the problem
\begin{quote}\begin{description}
\item[{Parameters}] \leavevmode
\sphinxstyleliteralstrong{\sphinxupquote{status\_code}} \textendash{} int

\end{description}\end{quote}

To specify the error code in the response. Like 400, 404, 500, etc.
\begin{quote}\begin{description}
\item[{Parameters}] \leavevmode
\sphinxstyleliteralstrong{\sphinxupquote{jsonify}} \textendash{} bool

\end{description}\end{quote}

Set the \sphinxcode{\sphinxupquote{jsonify}} attribute of the exception. The error handler can then 
check this value to decide how to convey the error to the user.
\index{to\_dict() (tiger\_leagues.models.exception.TigerLeaguesException method)@\spxentry{to\_dict()}\spxextra{tiger\_leagues.models.exception.TigerLeaguesException method}}

\begin{fulllineitems}
\phantomsection\label{\detokenize{tiger_leagues/models/readme:tiger_leagues.models.exception.TigerLeaguesException.to_dict}}\pysiglinewithargsret{\sphinxbfcode{\sphinxupquote{to\_dict}}}{}{}~\begin{quote}\begin{description}
\item[{Returns}] \leavevmode
\sphinxcode{\sphinxupquote{dict}}

\end{description}\end{quote}

A dict representation of the exception

\end{fulllineitems}


\end{fulllineitems}

\index{validate\_values() (in module tiger\_leagues.models.exception)@\spxentry{validate\_values()}\spxextra{in module tiger\_leagues.models.exception}}

\begin{fulllineitems}
\phantomsection\label{\detokenize{tiger_leagues/models/readme:tiger_leagues.models.exception.validate_values}}\pysiglinewithargsret{\sphinxcode{\sphinxupquote{tiger\_leagues.models.exception.}}\sphinxbfcode{\sphinxupquote{validate\_values}}}{\emph{data\_obj}, \emph{constraints}, \emph{jsonify=True}}{}
Helper function for validating JSON input
\begin{quote}\begin{description}
\item[{Parameters}] \leavevmode
\sphinxstyleliteralstrong{\sphinxupquote{data\_obj}} \textendash{} dict

\end{description}\end{quote}

A key-value pairing that needs to be validated
\begin{quote}\begin{description}
\item[{Parameters}] \leavevmode
\sphinxstyleliteralstrong{\sphinxupquote{constraints}} \textendash{} list{[}tuple{]}

\end{description}\end{quote}

Each tuple has 5 items. In order, they are: key (str), 
cast\_function (function), l\_limit (value), u\_limit (value), error\_msg (str)
\begin{quote}\begin{description}
\item[{Parameters}] \leavevmode
\sphinxstyleliteralstrong{\sphinxupquote{jsonify}} \textendash{} bool

\end{description}\end{quote}

If \sphinxcode{\sphinxupquote{True}}, the raised \sphinxcode{\sphinxupquote{TigerLeaguesException}} will have its jsonify 
attribute set.
\begin{quote}\begin{description}
\item[{Raises}] \leavevmode
\sphinxcode{\sphinxupquote{TigerLeaguesException}}

\end{description}\end{quote}

If any of the keys don’t exist or any of the values fail to meet the 
constraint.

\end{fulllineitems}



\chapter{Views}
\label{\detokenize{tiger_leagues/templates/readme:views}}\label{\detokenize{tiger_leagues/templates/readme:tiger-leagues-views}}\label{\detokenize{tiger_leagues/templates/readme::doc}}
As described on \sphinxhref{https://en.wikipedia.org/wiki/Model\%E2\%80\%93view\%E2\%80\%93controller\#Components}{Wikipedia},
a view is any output representation of information. In our case, our views
were all HTML files. We also have supporting stylesheets, images and JavaScript
in \sphinxcode{\sphinxupquote{./tiger\_leagues/static/*}}


\section{Design Decisions}
\label{\detokenize{tiger_leagues/templates/readme:design-decisions}}\label{\detokenize{tiger_leagues/templates/readme:views-design-decisions}}

\subsection{Uniform Design}
\label{\detokenize{tiger_leagues/templates/readme:uniform-design}}\label{\detokenize{tiger_leagues/templates/readme:id1}}
Our application has several pages, so it’s important to keep their design
uniform. Using the Jinja templates supported natively by Flask, we inherited
templates whenever we felt that a group of pages should share some design.


\section{Views Documentation}
\label{\detokenize{tiger_leagues/templates/readme:views-documentation}}\label{\detokenize{tiger_leagues/templates/readme:id2}}

\subsection{base.html}
\label{\detokenize{tiger_leagues/templates/readme:base-html}}
Serves as the overall template for all other HTML files. The header and
footer (and any other persistent content) should be added here.


\subsection{error.html}
\label{\detokenize{tiger_leagues/templates/readme:error-html}}
Used to render a custom error page.


\subsection{admin/*}
\label{\detokenize{tiger_leagues/templates/readme:admin}}
The HTML files found here correspond to different pages that are relevant to
admins, e.g.
\begin{itemize}
\item {} 
\sphinxcode{\sphinxupquote{admin\_league\_panel.html}} shows different actions that an admin can take

\item {} 
\sphinxcode{\sphinxupquote{manage\_members.html}} shows pages for managing league members.

\item {} 
\sphinxcode{\sphinxupquote{start\_league.html}} allows the admin to allocate league divisions and
generate fixtures.

\item {} 
\sphinxcode{\sphinxupquote{admin\_league\_homepage.html}} allows admins to approve pending scores.

\item {} 
\sphinxcode{\sphinxupquote{delete\_league.html}} allows admins to delete the league.

\end{itemize}


\subsection{auth/*}
\label{\detokenize{tiger_leagues/templates/readme:auth}}
Contains HTML files related to the authentication process, e.g.
\begin{itemize}
\item {} 
\sphinxcode{\sphinxupquote{login.html}} provides a link to Princeton’s Central Authentication System.

\end{itemize}

Since we’re using Princeton’s CAS, other auth-related pages such as resetting
a password, validating an email address, etc, are not necessary.


\subsection{league/*}
\label{\detokenize{tiger_leagues/templates/readme:league}}
Contains HTML files related to a league from the viewpoint of a non-admin
member.
\begin{itemize}
\item {} 
\sphinxcode{\sphinxupquote{browse.html}} shows leagues that a user can request to join.

\item {} 
\sphinxcode{\sphinxupquote{create\_league.html}} allows a user to create a new league.

\item {} 
\sphinxcode{\sphinxupquote{join\_league.html}} allows a user to request to join an existing league.

\item {} 
\sphinxcode{\sphinxupquote{update\_responses.html}} allows a user to update the responses that they
had submitted to the league.

\item {} 
\sphinxcode{\sphinxupquote{league\_base.html}} provides an inheritable template that has league header
information at the top.

\item {} 
\sphinxcode{\sphinxupquote{league\_homepage.html}} shows the current standings and upcoming matches of
the logged in user.

\item {} 
\sphinxcode{\sphinxupquote{member\_stats/league\_comparison\_base.html}} provides an inheritable
template for displaying a player(s) stats within a league.

\item {} 
\sphinxcode{\sphinxupquote{member\_stats/league\_side\_by\_side\_stats.html}} provides a side-by-side
comparison of the logged in user and any other comparable player.

\item {} 
\sphinxcode{\sphinxupquote{member\_stats/league\_single\_player\_stats.html}} provides league stats for a
single player (usually happens when user tries to view themselves, or a
player who is not in the same division)

\end{itemize}


\subsection{user/*}
\label{\detokenize{tiger_leagues/templates/readme:user}}
Contains HTML files related to a user’s account.
\begin{itemize}
\item {} 
\sphinxcode{\sphinxupquote{user\_profile.html}} allows a user to view and/or update their site-wide
profile.

\item {} 
\sphinxcode{\sphinxupquote{user\_notifications.html}} allows a user to read the notifications that
have been sent to their mailbox.

\end{itemize}


\chapter{Controllers}
\label{\detokenize{tiger_leagues/readme:controllers}}\label{\detokenize{tiger_leagues/readme:tiger-leagues-controllers}}\label{\detokenize{tiger_leagues/readme::doc}}
As described on \sphinxhref{https://en.wikipedia.org/wiki/Model\%E2\%80\%93view\%E2\%80\%93controller\#Components}{Wikipedia},
the contoller receives the input, optionally validates it and then passes
the input to the model.

Unlike the typical MVC model, we validated our most of our input in the models.
We did this because our test suite focused on the models.

Here is a quick breakdown of the input handled by the controllers:


\begin{savenotes}\sphinxattablestart
\centering
\begin{tabular}[t]{|*{2}{\X{1}{2}|}}
\hline
\sphinxstyletheadfamily 
Controller
&\sphinxstyletheadfamily 
Input to Relay to the Model
\\
\hline
{\hyperref[\detokenize{tiger_leagues/readme:module-tiger_leagues.auth}]{\sphinxcrossref{\sphinxcode{\sphinxupquote{tiger\_leagues.auth}}}}}

(This controller uses
{\hyperref[\detokenize{tiger_leagues/readme:module-tiger_leagues.cas_client}]{\sphinxcrossref{\sphinxcode{\sphinxupquote{tiger\_leagues.cas\_client}}}}} to
complete its tasks)
&\begin{itemize}
\item {} 
Results of CAS authentication for login

\item {} 
Request to log out the user

\end{itemize}
\\
\hline
{\hyperref[\detokenize{tiger_leagues/readme:module-tiger_leagues.league}]{\sphinxcrossref{\sphinxcode{\sphinxupquote{tiger\_leagues.league}}}}}
&\begin{itemize}
\item {} 
Creating a new league

\item {} 
Recording requests to join a league

\item {} 
Updating league standings

\item {} 
Fetching league standings

\item {} 
Fetching league matches

\item {} 
Fetching player stats

\item {} 
Processing score reports submitted by players

\item {} 
Processing player requests to leave a league

\end{itemize}
\\
\hline
{\hyperref[\detokenize{tiger_leagues/readme:module-tiger_leagues.admin}]{\sphinxcrossref{\sphinxcode{\sphinxupquote{tiger\_leagues.admin}}}}}

(This controller also checks that the
logged in user has admin privileges)
&\begin{itemize}
\item {} 
Adding/removing players from a league

\item {} 
Allocating league divisions

\item {} 
Processing score reports submitted by admins

\item {} 
Deleting a league

\end{itemize}
\\
\hline
{\hyperref[\detokenize{tiger_leagues/readme:module-tiger_leagues.user}]{\sphinxcrossref{\sphinxcode{\sphinxupquote{tiger\_leagues.user}}}}}
&\begin{itemize}
\item {} 
Fetch existing user profile

\item {} 
Update a user’s profile

\item {} 
Post notifications to a user

\item {} 
Read user’s notifications

\end{itemize}
\\
\hline
{\hyperref[\detokenize{tiger_leagues/readme:module-tiger_leagues.decorators}]{\sphinxcrossref{\sphinxcode{\sphinxupquote{tiger\_leagues.decorators}}}}}
&
Used as middleware
\begin{itemize}
\item {} 
Confirm that user is logged in

\item {} 
Refresh a user’s notifications

\end{itemize}
\\
\hline
\end{tabular}
\par
\sphinxattableend\end{savenotes}


\section{Design Decisions}
\label{\detokenize{tiger_leagues/readme:design-decisions}}\label{\detokenize{tiger_leagues/readme:controllers-design-decisions}}

\subsection{Use of Decorators}
\label{\detokenize{tiger_leagues/readme:use-of-decorators}}\label{\detokenize{tiger_leagues/readme:id1}}
We extensively used decorators, as defined in
{\hyperref[\detokenize{tiger_leagues/readme:module-tiger_leagues.decorators}]{\sphinxcrossref{\sphinxcode{\sphinxupquote{tiger\_leagues.decorators}}}}}, to enforce access control, e.g. only
logged in users can join a league, only admins can accept/reject league
members, etc.


\subsection{Graceful Error Handling}
\label{\detokenize{tiger_leagues/readme:graceful-error-handling}}\label{\detokenize{tiger_leagues/readme:id2}}
Since we defined a custom exception class,
{\hyperref[\detokenize{tiger_leagues/models/readme:module-tiger_leagues.models.exception}]{\sphinxcrossref{\sphinxcode{\sphinxupquote{tiger\_leagues.models.exception}}}}}, we were able to set an error handler
for any such exception. This allows us to gracefully show helpful error pages/
responses instead of the default ones.


\section{Controllers Documentation}
\label{\detokenize{tiger_leagues/readme:controllers-documentation}}\label{\detokenize{tiger_leagues/readme:id3}}

\subsection{tiger\_leagues.auth}
\label{\detokenize{tiger_leagues/readme:module-tiger_leagues.auth}}\label{\detokenize{tiger_leagues/readme:tiger-leagues-auth}}\index{tiger\_leagues.auth (module)@\spxentry{tiger\_leagues.auth}\spxextra{module}}
auth.py

Handles authentication-related requests e.g. \sphinxcode{\sphinxupquote{login}}, \sphinxcode{\sphinxupquote{logout}}.
Exposes a blueprint that handles requests made to the \sphinxtitleref{auth} endpoint
\index{cas\_login() (in module tiger\_leagues.auth)@\spxentry{cas\_login()}\spxextra{in module tiger\_leagues.auth}}

\begin{fulllineitems}
\phantomsection\label{\detokenize{tiger_leagues/readme:tiger_leagues.auth.cas_login}}\pysiglinewithargsret{\sphinxcode{\sphinxupquote{tiger\_leagues.auth.}}\sphinxbfcode{\sphinxupquote{cas\_login}}}{}{}
Log in users through CAS. At the end of the CAS-related stuff, the rest of 
the application expects to find a user object set in the session object.

Note that the contents of the session are public, but immutable. Please 
exclude values that you would not like the world to see. If sensitive data 
is needed, leave it to the caller to query the database themselves.
\begin{quote}\begin{description}
\item[{Returns}] \leavevmode
\sphinxcode{\sphinxupquote{flask.Response(code=302)}}

\end{description}\end{quote}

A redirect to the account creation page for new users or the homepage for 
any of the leagues that a returning user is associated with.

\end{fulllineitems}

\index{cas\_logout() (in module tiger\_leagues.auth)@\spxentry{cas\_logout()}\spxextra{in module tiger\_leagues.auth}}

\begin{fulllineitems}
\phantomsection\label{\detokenize{tiger_leagues/readme:tiger_leagues.auth.cas_logout}}\pysiglinewithargsret{\sphinxcode{\sphinxupquote{tiger\_leagues.auth.}}\sphinxbfcode{\sphinxupquote{cas\_logout}}}{}{}
Log out the currently logged in user.
\begin{quote}\begin{description}
\item[{Returns}] \leavevmode
\sphinxcode{\sphinxupquote{flask.Response(code=302)}}

\end{description}\end{quote}

Redirect to the login page.

\end{fulllineitems}

\index{index() (in module tiger\_leagues.auth)@\spxentry{index()}\spxextra{in module tiger\_leagues.auth}}

\begin{fulllineitems}
\phantomsection\label{\detokenize{tiger_leagues/readme:tiger_leagues.auth.index}}\pysiglinewithargsret{\sphinxcode{\sphinxupquote{tiger\_leagues.auth.}}\sphinxbfcode{\sphinxupquote{index}}}{}{}~\begin{quote}\begin{description}
\item[{Returns}] \leavevmode
\sphinxcode{\sphinxupquote{flask.Response(mimetype='text/HTML')}}

\end{description}\end{quote}

Render the login page if the person isn’t logged in, otherwise render a 
homepage for any of the leagues that they’re involved in.

\end{fulllineitems}



\subsection{tiger\_leagues.cas\_client}
\label{\detokenize{tiger_leagues/readme:module-tiger_leagues.cas_client}}\label{\detokenize{tiger_leagues/readme:tiger-leagues-cas-client}}\index{tiger\_leagues.cas\_client (module)@\spxentry{tiger\_leagues.cas\_client}\spxextra{module}}
cas\_client.py

A convenient wrapper around the central authentication system.

@authors: Scott Karlin, Alex Halderman, Brian Kernighan, Bob Dondero

@modified: Ported to Python 3.7 \& Flask by Chege Gitau
\index{CASClient (class in tiger\_leagues.cas\_client)@\spxentry{CASClient}\spxextra{class in tiger\_leagues.cas\_client}}

\begin{fulllineitems}
\phantomsection\label{\detokenize{tiger_leagues/readme:tiger_leagues.cas_client.CASClient}}\pysiglinewithargsret{\sphinxbfcode{\sphinxupquote{class }}\sphinxcode{\sphinxupquote{tiger\_leagues.cas\_client.}}\sphinxbfcode{\sphinxupquote{CASClient}}}{\emph{url='https://fed.princeton.edu/cas/'}}{}
A convenient wrapper around the central authentication system.
\index{authenticate() (tiger\_leagues.cas\_client.CASClient method)@\spxentry{authenticate()}\spxextra{tiger\_leagues.cas\_client.CASClient method}}

\begin{fulllineitems}
\phantomsection\label{\detokenize{tiger_leagues/readme:tiger_leagues.cas_client.CASClient.authenticate}}\pysiglinewithargsret{\sphinxbfcode{\sphinxupquote{authenticate}}}{\emph{request}, \emph{redirect}, \emph{session}}{}
Authenticate the remote user.
\begin{quote}\begin{description}
\item[{Parameters}] \leavevmode
\sphinxstyleliteralstrong{\sphinxupquote{request}} \textendash{} \sphinxcode{\sphinxupquote{flask.Request}}

\end{description}\end{quote}

A request that occurs as part of the CAS authentication process.
\begin{quote}\begin{description}
\item[{Parameters}] \leavevmode
\sphinxstyleliteralstrong{\sphinxupquote{redirect}} \textendash{} \sphinxcode{\sphinxupquote{flask.redirect}}

\end{description}\end{quote}

A function that, if called, returns a 3xx response
\begin{quote}\begin{description}
\item[{Parameters}] \leavevmode
\sphinxstyleliteralstrong{\sphinxupquote{session}} \textendash{} \sphinxcode{\sphinxupquote{flask.session}}

\end{description}\end{quote}

A session object whose values can be accessed by the rest of the 
application. If the authentication is successful, the \sphinxcode{\sphinxupquote{username}} 
attribute will be set.
\begin{quote}\begin{description}
\item[{Returns}] \leavevmode
\sphinxcode{\sphinxupquote{str}}

\end{description}\end{quote}

If the user has been successfully authenticated, return their username
\begin{quote}\begin{description}
\item[{Returns}] \leavevmode
\sphinxcode{\sphinxupquote{flask.Response(code=302)}}

\end{description}\end{quote}

If the user has not been successfully authenticated, redirect them to 
the CAS server’s login page.

\end{fulllineitems}

\index{strip\_ticket() (tiger\_leagues.cas\_client.CASClient method)@\spxentry{strip\_ticket()}\spxextra{tiger\_leagues.cas\_client.CASClient method}}

\begin{fulllineitems}
\phantomsection\label{\detokenize{tiger_leagues/readme:tiger_leagues.cas_client.CASClient.strip_ticket}}\pysiglinewithargsret{\sphinxbfcode{\sphinxupquote{strip\_ticket}}}{\emph{request}}{}~\begin{quote}\begin{description}
\item[{Parameters}] \leavevmode
\sphinxstyleliteralstrong{\sphinxupquote{request}} \textendash{} \sphinxcode{\sphinxupquote{flask.Request}}

\end{description}\end{quote}

A request that occurs as part of the CAS authentication process.
\begin{quote}\begin{description}
\item[{Returns}] \leavevmode
\sphinxcode{\sphinxupquote{str}}

\end{description}\end{quote}

The URL of the current request after stripping out the \sphinxtitleref{ticket} 
parameter added by the CAS server.

\end{fulllineitems}

\index{validate() (tiger\_leagues.cas\_client.CASClient method)@\spxentry{validate()}\spxextra{tiger\_leagues.cas\_client.CASClient method}}

\begin{fulllineitems}
\phantomsection\label{\detokenize{tiger_leagues/readme:tiger_leagues.cas_client.CASClient.validate}}\pysiglinewithargsret{\sphinxbfcode{\sphinxupquote{validate}}}{\emph{ticket}, \emph{request}}{}
Validate a login ticket by contacting the CAS server.
\begin{quote}\begin{description}
\item[{Parameters}] \leavevmode
\sphinxstyleliteralstrong{\sphinxupquote{request}} \textendash{} \sphinxcode{\sphinxupquote{str}}

\end{description}\end{quote}

A ticket that can be validated by CAS. Once a user authenticates 
themselves with CAS, CAS makes a GET request to the application. This 
GET request contains a ticket as one of its parameters.
\begin{quote}\begin{description}
\item[{Parameters}] \leavevmode
\sphinxstyleliteralstrong{\sphinxupquote{request}} \textendash{} \sphinxcode{\sphinxupquote{flask.Request}}

\end{description}\end{quote}

A request that occurs as part of the CAS authentication process.
\begin{quote}\begin{description}
\item[{Returns}] \leavevmode
\sphinxcode{\sphinxupquote{str}}

\end{description}\end{quote}

The user’s username if valid
\begin{quote}\begin{description}
\item[{Returns}] \leavevmode
\sphinxcode{\sphinxupquote{NoneType}}

\end{description}\end{quote}

Returned if the user is invalid

\end{fulllineitems}


\end{fulllineitems}



\subsection{tiger\_leagues.admin}
\label{\detokenize{tiger_leagues/readme:module-tiger_leagues.admin}}\label{\detokenize{tiger_leagues/readme:tiger-leagues-admin}}\index{tiger\_leagues.admin (module)@\spxentry{tiger\_leagues.admin}\spxextra{module}}
admin.py

Exposes a blueprint that handles requests made to \sphinxtitleref{/admin/*} endpoint.

The blueprint is then registered in the \sphinxcode{\sphinxupquote{\_\_init\_\_.py}} file and made available 
to the rest of the Flask application
\index{admin\_status\_required() (in module tiger\_leagues.admin)@\spxentry{admin\_status\_required()}\spxextra{in module tiger\_leagues.admin}}

\begin{fulllineitems}
\phantomsection\label{\detokenize{tiger_leagues/readme:tiger_leagues.admin.admin_status_required}}\pysiglinewithargsret{\sphinxcode{\sphinxupquote{tiger\_leagues.admin.}}\sphinxbfcode{\sphinxupquote{admin\_status\_required}}}{}{}
A decorator function that asserts that a user has admin privileges for the 
requested URL. This function is automatically called before any of the 
functions in the \sphinxcode{\sphinxupquote{admin}} module are executed. See 
\sphinxurl{http://flask.pocoo.org/docs/1.0/api/\#flask.Flask.before\_request}
\begin{quote}\begin{description}
\item[{Returns}] \leavevmode
\sphinxcode{\sphinxupquote{flask.Response(code=302)}}

\end{description}\end{quote}

A redirect to the login page if the user hasn’t logged in yet.
\begin{quote}\begin{description}
\item[{Returns}] \leavevmode
\sphinxcode{\sphinxupquote{flask.Response(code=302)}}

\end{description}\end{quote}

A redirect to an exception page if the user doesn’t have admin privileges in 
the league associated with this request.
\begin{quote}\begin{description}
\item[{Returns}] \leavevmode
\sphinxcode{\sphinxupquote{None}}

\end{description}\end{quote}

If the user has admin privileges for the current league, the request will 
then be passed on to the next function on the chain, typically the handler 
function for the request.

\end{fulllineitems}

\index{allocate\_league\_divisions() (in module tiger\_leagues.admin)@\spxentry{allocate\_league\_divisions()}\spxextra{in module tiger\_leagues.admin}}

\begin{fulllineitems}
\phantomsection\label{\detokenize{tiger_leagues/readme:tiger_leagues.admin.allocate_league_divisions}}\pysiglinewithargsret{\sphinxcode{\sphinxupquote{tiger\_leagues.admin.}}\sphinxbfcode{\sphinxupquote{allocate\_league\_divisions}}}{\emph{league\_id}}{}~\begin{quote}\begin{description}
\item[{Parameters}] \leavevmode
\sphinxstyleliteralstrong{\sphinxupquote{league\_id}} \textendash{} \sphinxcode{\sphinxupquote{int}}

\end{description}\end{quote}

The ID of the league associated with this request
\begin{quote}\begin{description}
\item[{Returns}] \leavevmode
\sphinxcode{\sphinxupquote{flask.Response(mimetype=application/json)}}

\end{description}\end{quote}

A JSON object containing allocations of players in a league into divisions

\end{fulllineitems}

\index{approve\_scores() (in module tiger\_leagues.admin)@\spxentry{approve\_scores()}\spxextra{in module tiger\_leagues.admin}}

\begin{fulllineitems}
\phantomsection\label{\detokenize{tiger_leagues/readme:tiger_leagues.admin.approve_scores}}\pysiglinewithargsret{\sphinxcode{\sphinxupquote{tiger\_leagues.admin.}}\sphinxbfcode{\sphinxupquote{approve\_scores}}}{\emph{league\_id}}{}~\begin{quote}\begin{description}
\item[{Parameters}] \leavevmode
\sphinxstyleliteralstrong{\sphinxupquote{league\_id}} \textendash{} \sphinxcode{\sphinxupquote{int}}

\end{description}\end{quote}

The ID of the league associated with this request
\begin{quote}\begin{description}
\item[{Returns}] \leavevmode
\sphinxcode{\sphinxupquote{flask.Response(mimetype=text/html)}}

\end{description}\end{quote}

If responding to a GET request, render a HTML page that allows the admin to 
approve any reported scores.
\begin{quote}\begin{description}
\item[{Returns}] \leavevmode
\sphinxcode{\sphinxupquote{flask.Response(mimetype=application/json)}}

\end{description}\end{quote}

If responding to a POST request, approve the scores as reported in the body 
of the POST request. Return a JSON object that confirms that the scores 
updated on the server.

\end{fulllineitems}

\index{delete\_league() (in module tiger\_leagues.admin)@\spxentry{delete\_league()}\spxextra{in module tiger\_leagues.admin}}

\begin{fulllineitems}
\phantomsection\label{\detokenize{tiger_leagues/readme:tiger_leagues.admin.delete_league}}\pysiglinewithargsret{\sphinxcode{\sphinxupquote{tiger\_leagues.admin.}}\sphinxbfcode{\sphinxupquote{delete\_league}}}{\emph{league\_id}}{}~\begin{quote}\begin{description}
\item[{Parameters}] \leavevmode
\sphinxstyleliteralstrong{\sphinxupquote{league\_id}} \textendash{} \sphinxcode{\sphinxupquote{int}}

\end{description}\end{quote}

The ID of the league associated with this request
\begin{quote}\begin{description}
\item[{Returns}] \leavevmode
\sphinxcode{\sphinxupquote{flask.Response(mimetype=text/html)}}

\end{description}\end{quote}

If responding to a GET request, render a HTML page that prompts the admin 
to delete the league, or abort the deletion
\begin{quote}\begin{description}
\item[{Returns}] \leavevmode
\sphinxcode{\sphinxupquote{flask.Response(mimetype=application/json)}}

\end{description}\end{quote}

If responding to a POST request, delete the league as specified in the POST 
request’s body. Return a JSON object that confirms that the league was 
indeed deleted from the server.

\end{fulllineitems}

\index{league\_has\_started() (in module tiger\_leagues.admin)@\spxentry{league\_has\_started()}\spxextra{in module tiger\_leagues.admin}}

\begin{fulllineitems}
\phantomsection\label{\detokenize{tiger_leagues/readme:tiger_leagues.admin.league_has_started}}\pysiglinewithargsret{\sphinxcode{\sphinxupquote{tiger\_leagues.admin.}}\sphinxbfcode{\sphinxupquote{league\_has\_started}}}{}{}
A decorator function that asserts that a league has already started. 
Called before approve\_scores and any other functions that should only take 
place with a started league.
\begin{quote}\begin{description}
\item[{Returns}] \leavevmode
\sphinxcode{\sphinxupquote{flask.Response(code=302)}}

\end{description}\end{quote}

A redirect to an exception page if the league has already started.
\begin{quote}\begin{description}
\item[{Returns}] \leavevmode
\sphinxcode{\sphinxupquote{None}}

\end{description}\end{quote}

If the league has not yet started, the request will 
then be passed on to the next function on the chain, typically the handler 
function for the request.

\end{fulllineitems}

\index{league\_homepage() (in module tiger\_leagues.admin)@\spxentry{league\_homepage()}\spxextra{in module tiger\_leagues.admin}}

\begin{fulllineitems}
\phantomsection\label{\detokenize{tiger_leagues/readme:tiger_leagues.admin.league_homepage}}\pysiglinewithargsret{\sphinxcode{\sphinxupquote{tiger\_leagues.admin.}}\sphinxbfcode{\sphinxupquote{league\_homepage}}}{\emph{league\_id}}{}~\begin{quote}\begin{description}
\item[{Parameters}] \leavevmode
\sphinxstyleliteralstrong{\sphinxupquote{league\_id}} \textendash{} \sphinxcode{\sphinxupquote{int}}

\end{description}\end{quote}

The ID of the league associated with this request
\begin{quote}\begin{description}
\item[{Returns}] \leavevmode
\sphinxcode{\sphinxupquote{flask.Response(mimetype='text/HTML')}}

\end{description}\end{quote}

Render a page with links to admin actions such as ‘Approve Members’

\end{fulllineitems}

\index{league\_not\_started() (in module tiger\_leagues.admin)@\spxentry{league\_not\_started()}\spxextra{in module tiger\_leagues.admin}}

\begin{fulllineitems}
\phantomsection\label{\detokenize{tiger_leagues/readme:tiger_leagues.admin.league_not_started}}\pysiglinewithargsret{\sphinxcode{\sphinxupquote{tiger\_leagues.admin.}}\sphinxbfcode{\sphinxupquote{league\_not\_started}}}{}{}
A decorator function that asserts that a league has not yet started. This 
function is automatically called before any of the functions in the 
\sphinxcode{\sphinxupquote{admin}} module are executed. See 
\sphinxurl{http://flask.pocoo.org/docs/1.0/api/\#flask.Flask.before\_request}
\begin{quote}\begin{description}
\item[{Returns}] \leavevmode
\sphinxcode{\sphinxupquote{flask.Response(code=302)}}

\end{description}\end{quote}

A redirect to an exception page if the league has already started.
\begin{quote}\begin{description}
\item[{Returns}] \leavevmode
\sphinxcode{\sphinxupquote{None}}

\end{description}\end{quote}

If the league has not yet started, the request will 
then be passed on to the next function on the chain, typically the handler 
function for the request.

\end{fulllineitems}

\index{league\_requests() (in module tiger\_leagues.admin)@\spxentry{league\_requests()}\spxextra{in module tiger\_leagues.admin}}

\begin{fulllineitems}
\phantomsection\label{\detokenize{tiger_leagues/readme:tiger_leagues.admin.league_requests}}\pysiglinewithargsret{\sphinxcode{\sphinxupquote{tiger\_leagues.admin.}}\sphinxbfcode{\sphinxupquote{league\_requests}}}{\emph{league\_id}}{}~\begin{quote}\begin{description}
\item[{Parameters}] \leavevmode
\sphinxstyleliteralstrong{\sphinxupquote{league\_id}} \textendash{} \sphinxcode{\sphinxupquote{int}}

\end{description}\end{quote}

The ID of the league associated with this request
\begin{quote}\begin{description}
\item[{Returns}] \leavevmode
\sphinxcode{\sphinxupquote{flask.Response(mimetype='text/HTML')}}

\end{description}\end{quote}

If responding to a GET request, render a template such that an admin can 
view the requests to join the league and can choose to accept or reject the 
join requests
\begin{quote}\begin{description}
\item[{Returns}] \leavevmode
\sphinxcode{\sphinxupquote{flask.Response(mimetype=application/json)}}

\end{description}\end{quote}

If responding to a POST request, update the join status of the users as 
instructed in the POST body. The JSON contains the keys \sphinxcode{\sphinxupquote{message}} and 
\sphinxcode{\sphinxupquote{success}}

\end{fulllineitems}

\index{manage\_members() (in module tiger\_leagues.admin)@\spxentry{manage\_members()}\spxextra{in module tiger\_leagues.admin}}

\begin{fulllineitems}
\phantomsection\label{\detokenize{tiger_leagues/readme:tiger_leagues.admin.manage_members}}\pysiglinewithargsret{\sphinxcode{\sphinxupquote{tiger\_leagues.admin.}}\sphinxbfcode{\sphinxupquote{manage\_members}}}{\emph{league\_id}}{}~\begin{quote}\begin{description}
\item[{Parameters}] \leavevmode
\sphinxstyleliteralstrong{\sphinxupquote{league\_id}} \textendash{} \sphinxcode{\sphinxupquote{int}}

\end{description}\end{quote}

The ID of the league associated with this request
\begin{quote}\begin{description}
\item[{Returns}] \leavevmode
\sphinxcode{\sphinxupquote{flask.Response(mimetype='text/HTML')}}

\end{description}\end{quote}

If responding to a GET request, render a template such that an admin can 
view the requests to join the league and can choose to accept or reject the 
join requests
\begin{quote}\begin{description}
\item[{Returns}] \leavevmode
\sphinxcode{\sphinxupquote{flask.Response(mimetype=application/json)}}

\end{description}\end{quote}

If responding to a POST request, update the join status of the users as 
instructed in the POST body. The JSON contains the keys \sphinxcode{\sphinxupquote{message}} and 
\sphinxcode{\sphinxupquote{success}}

\end{fulllineitems}

\index{start\_league() (in module tiger\_leagues.admin)@\spxentry{start\_league()}\spxextra{in module tiger\_leagues.admin}}

\begin{fulllineitems}
\phantomsection\label{\detokenize{tiger_leagues/readme:tiger_leagues.admin.start_league}}\pysiglinewithargsret{\sphinxcode{\sphinxupquote{tiger\_leagues.admin.}}\sphinxbfcode{\sphinxupquote{start\_league}}}{\emph{league\_id}}{}~\begin{quote}\begin{description}
\item[{Parameters}] \leavevmode
\sphinxstyleliteralstrong{\sphinxupquote{league\_id}} \textendash{} \sphinxcode{\sphinxupquote{int}}

\end{description}\end{quote}

The ID of the league associated with this request
\begin{quote}\begin{description}
\item[{Returns}] \leavevmode
\sphinxcode{\sphinxupquote{flask.Response(mimetype='text/HTML')}}

\end{description}\end{quote}

If responding to a GET request, render a template for setting the league 
configuration, e.g. frequency of matches
\begin{quote}\begin{description}
\item[{Returns}] \leavevmode
\sphinxcode{\sphinxupquote{flask.Response(mimetype=application/json)}}

\end{description}\end{quote}

If responding to a POST request, generate the league fixtures. Return a 
JSON response contains the keys \sphinxcode{\sphinxupquote{success}} and \sphinxcode{\sphinxupquote{message}}

\end{fulllineitems}



\subsection{tiger\_leagues.league}
\label{\detokenize{tiger_leagues/readme:module-tiger_leagues.league}}\label{\detokenize{tiger_leagues/readme:tiger-leagues-league}}\index{tiger\_leagues.league (module)@\spxentry{tiger\_leagues.league}\spxextra{module}}
league.py

Exposes a blueprint that handles requests made to \sphinxtitleref{/league/*} endpoint
\index{browse\_leagues() (in module tiger\_leagues.league)@\spxentry{browse\_leagues()}\spxextra{in module tiger\_leagues.league}}

\begin{fulllineitems}
\phantomsection\label{\detokenize{tiger_leagues/readme:tiger_leagues.league.browse_leagues}}\pysiglinewithargsret{\sphinxcode{\sphinxupquote{tiger\_leagues.league.}}\sphinxbfcode{\sphinxupquote{browse\_leagues}}}{}{}~\begin{quote}\begin{description}
\item[{Returns}] \leavevmode
\sphinxcode{\sphinxupquote{flask.Response(mimetype='text/html')}}

\end{description}\end{quote}

Render a page with a list of leagues that the user can request to join.

\end{fulllineitems}

\index{create\_league() (in module tiger\_leagues.league)@\spxentry{create\_league()}\spxextra{in module tiger\_leagues.league}}

\begin{fulllineitems}
\phantomsection\label{\detokenize{tiger_leagues/readme:tiger_leagues.league.create_league}}\pysiglinewithargsret{\sphinxcode{\sphinxupquote{tiger\_leagues.league.}}\sphinxbfcode{\sphinxupquote{create\_league}}}{}{}~\begin{quote}\begin{description}
\item[{Returns}] \leavevmode
\sphinxcode{\sphinxupquote{flask.Response(mimetype='text/html')}}

\end{description}\end{quote}

If responding to a GET request, render a template that can be used to 
create a new league.
\begin{quote}\begin{description}
\item[{Returns}] \leavevmode
\sphinxcode{\sphinxupquote{flask.Response(mimetype='application/json')}}

\end{description}\end{quote}

If responding to a POST request, return a JSON object confirming whether 
the league was created. The JSON sent in the POST request should have these 
keys: \sphinxcode{\sphinxupquote{league\_name, description, points\_per\_win, points\_per\_draw, 
points\_per\_loss, registration\_deadline and additional\_questions}}

\end{fulllineitems}

\index{index() (in module tiger\_leagues.league)@\spxentry{index()}\spxextra{in module tiger\_leagues.league}}

\begin{fulllineitems}
\phantomsection\label{\detokenize{tiger_leagues/readme:tiger_leagues.league.index}}\pysiglinewithargsret{\sphinxcode{\sphinxupquote{tiger\_leagues.league.}}\sphinxbfcode{\sphinxupquote{index}}}{}{}
A user that already has an account will be redirected here. The user details 
will be present in the session object.
\begin{quote}\begin{description}
\item[{Returns}] \leavevmode
\sphinxcode{\sphinxupquote{flask.Response(mimetype='text/HTML')}}

\end{description}\end{quote}

If the user is a part of any leagues, render that league’s homepage.
\begin{quote}\begin{description}
\item[{Returns}] \leavevmode
\sphinxcode{\sphinxupquote{flask.Response(code=302)}}

\end{description}\end{quote}

If the user is not a part of any league, redirect them to a page that allows 
them to browse available leagues.

\end{fulllineitems}

\index{join\_league() (in module tiger\_leagues.league)@\spxentry{join\_league()}\spxextra{in module tiger\_leagues.league}}

\begin{fulllineitems}
\phantomsection\label{\detokenize{tiger_leagues/readme:tiger_leagues.league.join_league}}\pysiglinewithargsret{\sphinxcode{\sphinxupquote{tiger\_leagues.league.}}\sphinxbfcode{\sphinxupquote{join\_league}}}{\emph{league\_id}}{}~\begin{quote}\begin{description}
\item[{Parameters}] \leavevmode
\sphinxstyleliteralstrong{\sphinxupquote{league\_id}} \textendash{} \sphinxcode{\sphinxupquote{int}}

\end{description}\end{quote}

The ID of the league associated with this request
\begin{quote}\begin{description}
\item[{Returns}] \leavevmode
\sphinxcode{\sphinxupquote{flask.Response(mimetype='text/html')}}

\end{description}\end{quote}

Render the form that needs to be filled by users that wish to join 
this league.
\begin{quote}\begin{description}
\item[{Returns}] \leavevmode
\sphinxcode{\sphinxupquote{flask.Response(mimetype='text/html')}}

\end{description}\end{quote}

Process the form submitted by the user who wants to join this league. Return 
a JSON object that confirms the status of the join request.

\end{fulllineitems}

\index{league\_homepage() (in module tiger\_leagues.league)@\spxentry{league\_homepage()}\spxextra{in module tiger\_leagues.league}}

\begin{fulllineitems}
\phantomsection\label{\detokenize{tiger_leagues/readme:tiger_leagues.league.league_homepage}}\pysiglinewithargsret{\sphinxcode{\sphinxupquote{tiger\_leagues.league.}}\sphinxbfcode{\sphinxupquote{league\_homepage}}}{\emph{league\_id}}{}~\begin{quote}\begin{description}
\item[{Parameters}] \leavevmode
\sphinxstyleliteralstrong{\sphinxupquote{league\_id}} \textendash{} \sphinxcode{\sphinxupquote{int}}

\end{description}\end{quote}

The ID of the league associated with this request
\begin{quote}\begin{description}
\item[{Returns}] \leavevmode
\sphinxcode{\sphinxupquote{flask.Response(mimetype='text/html')}}

\end{description}\end{quote}

Render a template for the provided league and the associated user. The 
template includes information such as \sphinxcode{\sphinxupquote{standings, media\_feed, score\_reports, 
upcoming\_matches}}, etc.

\end{fulllineitems}

\index{league\_member() (in module tiger\_leagues.league)@\spxentry{league\_member()}\spxextra{in module tiger\_leagues.league}}

\begin{fulllineitems}
\phantomsection\label{\detokenize{tiger_leagues/readme:tiger_leagues.league.league_member}}\pysiglinewithargsret{\sphinxcode{\sphinxupquote{tiger\_leagues.league.}}\sphinxbfcode{\sphinxupquote{league\_member}}}{\emph{league\_id}, \emph{other\_user\_id}}{}~\begin{quote}\begin{description}
\item[{Parameters}] \leavevmode
\sphinxstyleliteralstrong{\sphinxupquote{league\_id}} \textendash{} \sphinxcode{\sphinxupquote{int}}

\end{description}\end{quote}

The ID of the league associated with this request
\begin{quote}\begin{description}
\item[{Parameters}] \leavevmode
\sphinxstyleliteralstrong{\sphinxupquote{other\_user\_id}} \textendash{} \sphinxcode{\sphinxupquote{int}}

\end{description}\end{quote}

The ID of the user whose data should be fetched.
\begin{quote}\begin{description}
\item[{Returns}] \leavevmode
\sphinxcode{\sphinxupquote{flask.Response(mimetype='text/html')}}

\end{description}\end{quote}

If \sphinxcode{\sphinxupquote{other\_user\_id == current\_user\_id}}, the page shows the currently 
logged in user’s stats.

If \sphinxcode{\sphinxupquote{other\_user\_id}} does not belong to the same division as the currently 
logged in user, the stats of this other player are shown.

If the two users are in the same division, then the page shows both player’s 
stats in the league in a side-by-side fashion.

\end{fulllineitems}

\index{leave\_league() (in module tiger\_leagues.league)@\spxentry{leave\_league()}\spxextra{in module tiger\_leagues.league}}

\begin{fulllineitems}
\phantomsection\label{\detokenize{tiger_leagues/readme:tiger_leagues.league.leave_league}}\pysiglinewithargsret{\sphinxcode{\sphinxupquote{tiger\_leagues.league.}}\sphinxbfcode{\sphinxupquote{leave\_league}}}{\emph{league\_id}}{}~\begin{quote}\begin{description}
\item[{Parameters}] \leavevmode
\sphinxstyleliteralstrong{\sphinxupquote{league\_id}} \textendash{} \sphinxcode{\sphinxupquote{int}}

\end{description}\end{quote}

The ID of the league associated with this request
\begin{quote}\begin{description}
\item[{Returns}] \leavevmode
\sphinxcode{\sphinxupquote{flask.Response(mimetype='application/json')}}

\end{description}\end{quote}

The JSON object contains a confirmation that the user was removed from the 
league.

\end{fulllineitems}

\index{process\_score\_submit() (in module tiger\_leagues.league)@\spxentry{process\_score\_submit()}\spxextra{in module tiger\_leagues.league}}

\begin{fulllineitems}
\phantomsection\label{\detokenize{tiger_leagues/readme:tiger_leagues.league.process_score_submit}}\pysiglinewithargsret{\sphinxcode{\sphinxupquote{tiger\_leagues.league.}}\sphinxbfcode{\sphinxupquote{process\_score\_submit}}}{\emph{league\_id}}{}
Persist the score submitted by the user. The body of the POST object should 
have the following keys: \sphinxtitleref{my\_score, opponent\_score, match\_id}
\begin{quote}\begin{description}
\item[{Parameters}] \leavevmode
\sphinxstyleliteralstrong{\sphinxupquote{league\_id}} \textendash{} \sphinxcode{\sphinxupquote{int}}

\end{description}\end{quote}

The ID of the league associated with this request
\begin{quote}\begin{description}
\item[{Returns}] \leavevmode
\sphinxcode{\sphinxupquote{flask.Response(mimetype='application/json')}}

\end{description}\end{quote}

The JSON object contains the keys \sphinxcode{\sphinxupquote{success}} and \sphinxcode{\sphinxupquote{message}} whose values 
set appropriately.

\end{fulllineitems}

\index{update\_responses() (in module tiger\_leagues.league)@\spxentry{update\_responses()}\spxextra{in module tiger\_leagues.league}}

\begin{fulllineitems}
\phantomsection\label{\detokenize{tiger_leagues/readme:tiger_leagues.league.update_responses}}\pysiglinewithargsret{\sphinxcode{\sphinxupquote{tiger\_leagues.league.}}\sphinxbfcode{\sphinxupquote{update\_responses}}}{\emph{league\_id}}{}~\begin{quote}\begin{description}
\item[{Parameters}] \leavevmode
\sphinxstyleliteralstrong{\sphinxupquote{league\_id}} \textendash{} \sphinxcode{\sphinxupquote{int}}

\end{description}\end{quote}

The ID of the league associated with this request
\begin{quote}\begin{description}
\item[{Returns}] \leavevmode
\sphinxcode{\sphinxupquote{flask.Response(mimetype='text/html')}}

\end{description}\end{quote}

Render the form that needs to be filled by users that wish to update their responses 
to league-specific questions.
\begin{quote}\begin{description}
\item[{Returns}] \leavevmode
\sphinxcode{\sphinxupquote{flask.Response(mimetype='text/html')}}

\end{description}\end{quote}

Process the form submitted by the user. Return 
a JSON object that confirms the status of the submission.

\end{fulllineitems}



\subsection{tiger\_leagues.user}
\label{\detokenize{tiger_leagues/readme:module-tiger_leagues.user}}\label{\detokenize{tiger_leagues/readme:tiger-leagues-user}}\index{tiger\_leagues.user (module)@\spxentry{tiger\_leagues.user}\spxextra{module}}
user.py

Exposes a blueprint that handles requests made to \sphinxtitleref{/user/*} endpoint
\index{display\_user\_profile() (in module tiger\_leagues.user)@\spxentry{display\_user\_profile()}\spxextra{in module tiger\_leagues.user}}

\begin{fulllineitems}
\phantomsection\label{\detokenize{tiger_leagues/readme:tiger_leagues.user.display_user_profile}}\pysiglinewithargsret{\sphinxcode{\sphinxupquote{tiger\_leagues.user.}}\sphinxbfcode{\sphinxupquote{display\_user\_profile}}}{}{}~\begin{quote}\begin{description}
\item[{Returns}] \leavevmode
\sphinxcode{\sphinxupquote{flask.Response(mimetype-'text/html')}}

\end{description}\end{quote}

Render a template that contains user information such as: \sphinxcode{\sphinxupquote{net\_id, 
preferred\_name, preferred\_email, phone\_number, room\_number, 
associated\_leagues}}

\end{fulllineitems}

\index{modify\_notification\_status() (in module tiger\_leagues.user)@\spxentry{modify\_notification\_status()}\spxextra{in module tiger\_leagues.user}}

\begin{fulllineitems}
\phantomsection\label{\detokenize{tiger_leagues/readme:tiger_leagues.user.modify_notification_status}}\pysiglinewithargsret{\sphinxcode{\sphinxupquote{tiger\_leagues.user.}}\sphinxbfcode{\sphinxupquote{modify\_notification\_status}}}{}{}~\begin{quote}\begin{description}
\item[{Returns}] \leavevmode
\sphinxcode{\sphinxupquote{flask.Response(mimetype-'application/json')}}

\end{description}\end{quote}

The JSON object is keyed by \sphinxcode{\sphinxupquote{success}} and \sphinxcode{\sphinxupquote{message}}

\end{fulllineitems}

\index{update\_user\_profile() (in module tiger\_leagues.user)@\spxentry{update\_user\_profile()}\spxextra{in module tiger\_leagues.user}}

\begin{fulllineitems}
\phantomsection\label{\detokenize{tiger_leagues/readme:tiger_leagues.user.update_user_profile}}\pysiglinewithargsret{\sphinxcode{\sphinxupquote{tiger\_leagues.user.}}\sphinxbfcode{\sphinxupquote{update\_user\_profile}}}{}{}~\begin{quote}\begin{description}
\item[{Returns}] \leavevmode
\sphinxcode{\sphinxupquote{flask.Response(mimetype-'text/html')}}

\end{description}\end{quote}

Update the information stored about a user. Render a template that contains 
user information such as: \sphinxcode{\sphinxupquote{net\_id, preferred\_name, preferred\_email, 
phone\_number, room\_number, associated\_leagues}}

\end{fulllineitems}

\index{view\_notifications() (in module tiger\_leagues.user)@\spxentry{view\_notifications()}\spxextra{in module tiger\_leagues.user}}

\begin{fulllineitems}
\phantomsection\label{\detokenize{tiger_leagues/readme:tiger_leagues.user.view_notifications}}\pysiglinewithargsret{\sphinxcode{\sphinxupquote{tiger\_leagues.user.}}\sphinxbfcode{\sphinxupquote{view\_notifications}}}{}{}~\begin{quote}\begin{description}
\item[{Returns}] \leavevmode
\sphinxcode{\sphinxupquote{flask.Response(mimetype-'text/html')}}

\end{description}\end{quote}

Render the user’s pending messages

\end{fulllineitems}



\subsection{tiger\_leagues.decorators}
\label{\detokenize{tiger_leagues/readme:module-tiger_leagues.decorators}}\label{\detokenize{tiger_leagues/readme:tiger-leagues-decorators}}\index{tiger\_leagues.decorators (module)@\spxentry{tiger\_leagues.decorators}\spxextra{module}}
decorators.py

A decorator is a function that wraps and replaces another function. If there’s 
a functionality that you wish to extend to multiple functions, you should 
probably add the functionality as a decorator.

\sphinxurl{http://flask.pocoo.org/docs/1.0/patterns/viewdecorators/}
\index{login\_required() (in module tiger\_leagues.decorators)@\spxentry{login\_required()}\spxextra{in module tiger\_leagues.decorators}}

\begin{fulllineitems}
\phantomsection\label{\detokenize{tiger_leagues/readme:tiger_leagues.decorators.login_required}}\pysiglinewithargsret{\sphinxcode{\sphinxupquote{tiger\_leagues.decorators.}}\sphinxbfcode{\sphinxupquote{login\_required}}}{\emph{f}}{}
A decorator function that is used to confirm that a user is logged in before 
viewing/using certain URLs.

\sphinxurl{http://flask.pocoo.org/docs/1.0/patterns/viewdecorators/\#login-required-decorator}
\begin{quote}\begin{description}
\item[{Parameters}] \leavevmode
\sphinxstyleliteralstrong{\sphinxupquote{f}} \textendash{} \sphinxcode{\sphinxupquote{function}}

\end{description}\end{quote}

A function that should be accessed only by authenticated users.
\begin{quote}\begin{description}
\item[{Returns}] \leavevmode
\sphinxcode{\sphinxupquote{flask.Response(code=302)}}

\end{description}\end{quote}

If the user isn’t logged in, redirect them to the application’s login page.
\begin{quote}\begin{description}
\item[{Returns}] \leavevmode
\sphinxcode{\sphinxupquote{function}}

\end{description}\end{quote}

If the user is logged in, return a function that is equal to the one 
that was passed as a parameter

\end{fulllineitems}

\index{refresh\_user\_profile() (in module tiger\_leagues.decorators)@\spxentry{refresh\_user\_profile()}\spxextra{in module tiger\_leagues.decorators}}

\begin{fulllineitems}
\phantomsection\label{\detokenize{tiger_leagues/readme:tiger_leagues.decorators.refresh_user_profile}}\pysiglinewithargsret{\sphinxcode{\sphinxupquote{tiger\_leagues.decorators.}}\sphinxbfcode{\sphinxupquote{refresh\_user\_profile}}}{\emph{f}}{}
A decorator function that is updates the user object stored in the session 
object. This is helpful when keeping the user up to date.

\sphinxurl{http://flask.pocoo.org/docs/1.0/patterns/viewdecorators/\#login-required-decorator}
\begin{quote}\begin{description}
\item[{Parameters}] \leavevmode
\sphinxstyleliteralstrong{\sphinxupquote{f}} \textendash{} \sphinxcode{\sphinxupquote{function}}

\end{description}\end{quote}

A function that would benefit from an updated user object
\begin{quote}\begin{description}
\item[{Returns}] \leavevmode
\sphinxcode{\sphinxupquote{function}}

\end{description}\end{quote}

The incoming function is always returned as updating the user object is a 
side effect.

\end{fulllineitems}



\subsection{tiger\_leagues.wsgi}
\label{\detokenize{tiger_leagues/readme:module-tiger_leagues.wsgi}}\label{\detokenize{tiger_leagues/readme:tiger-leagues-wsgi}}\index{tiger\_leagues.wsgi (module)@\spxentry{tiger\_leagues.wsgi}\spxextra{module}}
wsgi.py

Expose Flask application object as the WSGI application. The WSGI app will then
be ran by a WSGI server. Flask’s built-in server is not suitable for production.

In our case, in \sphinxcode{\sphinxupquote{../Procfile}}, we ask \sphinxcode{\sphinxupquote{gunicorn}} to use the \sphinxcode{\sphinxupquote{app}} object 
that is exposed in the \sphinxcode{\sphinxupquote{tiger\_leagues}} module.

\sphinxurl{http://flask.pocoo.org/docs/1.0/deploying/}


\chapter{Indices and tables}
\label{\detokenize{index:indices-and-tables}}\begin{itemize}
\item {} 
\DUrole{xref,std,std-ref}{genindex}

\item {} 
\DUrole{xref,std,std-ref}{modindex}

\item {} 
\DUrole{xref,std,std-ref}{search}

\end{itemize}


\renewcommand{\indexname}{Python Module Index}
\begin{sphinxtheindex}
\let\bigletter\sphinxstyleindexlettergroup
\bigletter{t}
\item\relax\sphinxstyleindexentry{tiger\_leagues.admin}\sphinxstyleindexpageref{tiger_leagues/readme:\detokenize{module-tiger_leagues.admin}}
\item\relax\sphinxstyleindexentry{tiger\_leagues.auth}\sphinxstyleindexpageref{tiger_leagues/readme:\detokenize{module-tiger_leagues.auth}}
\item\relax\sphinxstyleindexentry{tiger\_leagues.cas\_client}\sphinxstyleindexpageref{tiger_leagues/readme:\detokenize{module-tiger_leagues.cas_client}}
\item\relax\sphinxstyleindexentry{tiger\_leagues.decorators}\sphinxstyleindexpageref{tiger_leagues/readme:\detokenize{module-tiger_leagues.decorators}}
\item\relax\sphinxstyleindexentry{tiger\_leagues.league}\sphinxstyleindexpageref{tiger_leagues/readme:\detokenize{module-tiger_leagues.league}}
\item\relax\sphinxstyleindexentry{tiger\_leagues.models.admin\_model}\sphinxstyleindexpageref{tiger_leagues/models/readme:\detokenize{module-tiger_leagues.models.admin_model}}
\item\relax\sphinxstyleindexentry{tiger\_leagues.models.config}\sphinxstyleindexpageref{tiger_leagues/models/readme:\detokenize{module-tiger_leagues.models.config}}
\item\relax\sphinxstyleindexentry{tiger\_leagues.models.db\_model}\sphinxstyleindexpageref{tiger_leagues/models/readme:\detokenize{module-tiger_leagues.models.db_model}}
\item\relax\sphinxstyleindexentry{tiger\_leagues.models.exception}\sphinxstyleindexpageref{tiger_leagues/models/readme:\detokenize{module-tiger_leagues.models.exception}}
\item\relax\sphinxstyleindexentry{tiger\_leagues.models.league\_model}\sphinxstyleindexpageref{tiger_leagues/models/readme:\detokenize{module-tiger_leagues.models.league_model}}
\item\relax\sphinxstyleindexentry{tiger\_leagues.models.user\_model}\sphinxstyleindexpageref{tiger_leagues/models/readme:\detokenize{module-tiger_leagues.models.user_model}}
\item\relax\sphinxstyleindexentry{tiger\_leagues.user}\sphinxstyleindexpageref{tiger_leagues/readme:\detokenize{module-tiger_leagues.user}}
\item\relax\sphinxstyleindexentry{tiger\_leagues.wsgi}\sphinxstyleindexpageref{tiger_leagues/readme:\detokenize{module-tiger_leagues.wsgi}}
\end{sphinxtheindex}

\renewcommand{\indexname}{Index}
\printindex
\end{document}